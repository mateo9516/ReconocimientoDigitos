\documentclass[11pt]{article}

    \usepackage[breakable]{tcolorbox}
    \usepackage{parskip} % Stop auto-indenting (to mimic markdown behaviour)
    

    % Basic figure setup, for now with no caption control since it's done
    % automatically by Pandoc (which extracts ![](path) syntax from Markdown).
    \usepackage{graphicx}
    % Maintain compatibility with old templates. Remove in nbconvert 6.0
    \let\Oldincludegraphics\includegraphics
    % Ensure that by default, figures have no caption (until we provide a
    % proper Figure object with a Caption API and a way to capture that
    % in the conversion process - todo).
    \usepackage{caption}
    \DeclareCaptionFormat{nocaption}{}
    \captionsetup{format=nocaption,aboveskip=0pt,belowskip=0pt}

    \usepackage{float}
    \floatplacement{figure}{H} % forces figures to be placed at the correct location
    \usepackage{xcolor} % Allow colors to be defined
    \usepackage{enumerate} % Needed for markdown enumerations to work
    \usepackage{geometry} % Used to adjust the document margins
    \usepackage{amsmath} % Equations
    \usepackage{amssymb} % Equations
    \usepackage{textcomp} % defines textquotesingle
    % Hack from http://tex.stackexchange.com/a/47451/13684:
    \AtBeginDocument{%
        \def\PYZsq{\textquotesingle}% Upright quotes in Pygmentized code
    }
    \usepackage{upquote} % Upright quotes for verbatim code
    \usepackage{eurosym} % defines \euro

    \usepackage{iftex}
    \ifPDFTeX
        \usepackage[T1]{fontenc}
        \IfFileExists{alphabeta.sty}{
              \usepackage{alphabeta}
          }{
              \usepackage[mathletters]{ucs}
              \usepackage[utf8x]{inputenc}
          }
    \else
        \usepackage{fontspec}
        \usepackage{unicode-math}
    \fi

    \usepackage{fancyvrb} % verbatim replacement that allows latex
    \usepackage{grffile} % extends the file name processing of package graphics
                         % to support a larger range
    \makeatletter % fix for old versions of grffile with XeLaTeX
    \@ifpackagelater{grffile}{2019/11/01}
    {
      % Do nothing on new versions
    }
    {
      \def\Gread@@xetex#1{%
        \IfFileExists{"\Gin@base".bb}%
        {\Gread@eps{\Gin@base.bb}}%
        {\Gread@@xetex@aux#1}%
      }
    }
    \makeatother
    \usepackage[Export]{adjustbox} % Used to constrain images to a maximum size
    \adjustboxset{max size={0.9\linewidth}{0.9\paperheight}}

    % The hyperref package gives us a pdf with properly built
    % internal navigation ('pdf bookmarks' for the table of contents,
    % internal cross-reference links, web links for URLs, etc.)
    \usepackage{hyperref}
    % The default LaTeX title has an obnoxious amount of whitespace. By default,
    % titling removes some of it. It also provides customization options.
    \usepackage{titling}
    \usepackage{longtable} % longtable support required by pandoc >1.10
    \usepackage{booktabs}  % table support for pandoc > 1.12.2
    \usepackage{array}     % table support for pandoc >= 2.11.3
    \usepackage{calc}      % table minipage width calculation for pandoc >= 2.11.1
    \usepackage[inline]{enumitem} % IRkernel/repr support (it uses the enumerate* environment)
    \usepackage[normalem]{ulem} % ulem is needed to support strikethroughs (\sout)
                                % normalem makes italics be italics, not underlines
    \usepackage{mathrsfs}
    

    
    % Colors for the hyperref package
    \definecolor{urlcolor}{rgb}{0,.145,.698}
    \definecolor{linkcolor}{rgb}{.71,0.21,0.01}
    \definecolor{citecolor}{rgb}{.12,.54,.11}

    % ANSI colors
    \definecolor{ansi-black}{HTML}{3E424D}
    \definecolor{ansi-black-intense}{HTML}{282C36}
    \definecolor{ansi-red}{HTML}{E75C58}
    \definecolor{ansi-red-intense}{HTML}{B22B31}
    \definecolor{ansi-green}{HTML}{00A250}
    \definecolor{ansi-green-intense}{HTML}{007427}
    \definecolor{ansi-yellow}{HTML}{DDB62B}
    \definecolor{ansi-yellow-intense}{HTML}{B27D12}
    \definecolor{ansi-blue}{HTML}{208FFB}
    \definecolor{ansi-blue-intense}{HTML}{0065CA}
    \definecolor{ansi-magenta}{HTML}{D160C4}
    \definecolor{ansi-magenta-intense}{HTML}{A03196}
    \definecolor{ansi-cyan}{HTML}{60C6C8}
    \definecolor{ansi-cyan-intense}{HTML}{258F8F}
    \definecolor{ansi-white}{HTML}{C5C1B4}
    \definecolor{ansi-white-intense}{HTML}{A1A6B2}
    \definecolor{ansi-default-inverse-fg}{HTML}{FFFFFF}
    \definecolor{ansi-default-inverse-bg}{HTML}{000000}

    % common color for the border for error outputs.
    \definecolor{outerrorbackground}{HTML}{FFDFDF}

    % commands and environments needed by pandoc snippets
    % extracted from the output of `pandoc -s`
    \providecommand{\tightlist}{%
      \setlength{\itemsep}{0pt}\setlength{\parskip}{0pt}}
    \DefineVerbatimEnvironment{Highlighting}{Verbatim}{commandchars=\\\{\}}
    % Add ',fontsize=\small' for more characters per line
    \newenvironment{Shaded}{}{}
    \newcommand{\KeywordTok}[1]{\textcolor[rgb]{0.00,0.44,0.13}{\textbf{{#1}}}}
    \newcommand{\DataTypeTok}[1]{\textcolor[rgb]{0.56,0.13,0.00}{{#1}}}
    \newcommand{\DecValTok}[1]{\textcolor[rgb]{0.25,0.63,0.44}{{#1}}}
    \newcommand{\BaseNTok}[1]{\textcolor[rgb]{0.25,0.63,0.44}{{#1}}}
    \newcommand{\FloatTok}[1]{\textcolor[rgb]{0.25,0.63,0.44}{{#1}}}
    \newcommand{\CharTok}[1]{\textcolor[rgb]{0.25,0.44,0.63}{{#1}}}
    \newcommand{\StringTok}[1]{\textcolor[rgb]{0.25,0.44,0.63}{{#1}}}
    \newcommand{\CommentTok}[1]{\textcolor[rgb]{0.38,0.63,0.69}{\textit{{#1}}}}
    \newcommand{\OtherTok}[1]{\textcolor[rgb]{0.00,0.44,0.13}{{#1}}}
    \newcommand{\AlertTok}[1]{\textcolor[rgb]{1.00,0.00,0.00}{\textbf{{#1}}}}
    \newcommand{\FunctionTok}[1]{\textcolor[rgb]{0.02,0.16,0.49}{{#1}}}
    \newcommand{\RegionMarkerTok}[1]{{#1}}
    \newcommand{\ErrorTok}[1]{\textcolor[rgb]{1.00,0.00,0.00}{\textbf{{#1}}}}
    \newcommand{\NormalTok}[1]{{#1}}

    % Additional commands for more recent versions of Pandoc
    \newcommand{\ConstantTok}[1]{\textcolor[rgb]{0.53,0.00,0.00}{{#1}}}
    \newcommand{\SpecialCharTok}[1]{\textcolor[rgb]{0.25,0.44,0.63}{{#1}}}
    \newcommand{\VerbatimStringTok}[1]{\textcolor[rgb]{0.25,0.44,0.63}{{#1}}}
    \newcommand{\SpecialStringTok}[1]{\textcolor[rgb]{0.73,0.40,0.53}{{#1}}}
    \newcommand{\ImportTok}[1]{{#1}}
    \newcommand{\DocumentationTok}[1]{\textcolor[rgb]{0.73,0.13,0.13}{\textit{{#1}}}}
    \newcommand{\AnnotationTok}[1]{\textcolor[rgb]{0.38,0.63,0.69}{\textbf{\textit{{#1}}}}}
    \newcommand{\CommentVarTok}[1]{\textcolor[rgb]{0.38,0.63,0.69}{\textbf{\textit{{#1}}}}}
    \newcommand{\VariableTok}[1]{\textcolor[rgb]{0.10,0.09,0.49}{{#1}}}
    \newcommand{\ControlFlowTok}[1]{\textcolor[rgb]{0.00,0.44,0.13}{\textbf{{#1}}}}
    \newcommand{\OperatorTok}[1]{\textcolor[rgb]{0.40,0.40,0.40}{{#1}}}
    \newcommand{\BuiltInTok}[1]{{#1}}
    \newcommand{\ExtensionTok}[1]{{#1}}
    \newcommand{\PreprocessorTok}[1]{\textcolor[rgb]{0.74,0.48,0.00}{{#1}}}
    \newcommand{\AttributeTok}[1]{\textcolor[rgb]{0.49,0.56,0.16}{{#1}}}
    \newcommand{\InformationTok}[1]{\textcolor[rgb]{0.38,0.63,0.69}{\textbf{\textit{{#1}}}}}
    \newcommand{\WarningTok}[1]{\textcolor[rgb]{0.38,0.63,0.69}{\textbf{\textit{{#1}}}}}


    % Define a nice break command that doesn't care if a line doesn't already
    % exist.
    \def\br{\hspace*{\fill} \\* }
    % Math Jax compatibility definitions
    \def\gt{>}
    \def\lt{<}
    \let\Oldtex\TeX
    \let\Oldlatex\LaTeX
    \renewcommand{\TeX}{\textrm{\Oldtex}}
    \renewcommand{\LaTeX}{\textrm{\Oldlatex}}
    % Document parameters
    % Document title
    \title{semana5\_Echeverry\_Mateo}
    
    
    
    
    
% Pygments definitions
\makeatletter
\def\PY@reset{\let\PY@it=\relax \let\PY@bf=\relax%
    \let\PY@ul=\relax \let\PY@tc=\relax%
    \let\PY@bc=\relax \let\PY@ff=\relax}
\def\PY@tok#1{\csname PY@tok@#1\endcsname}
\def\PY@toks#1+{\ifx\relax#1\empty\else%
    \PY@tok{#1}\expandafter\PY@toks\fi}
\def\PY@do#1{\PY@bc{\PY@tc{\PY@ul{%
    \PY@it{\PY@bf{\PY@ff{#1}}}}}}}
\def\PY#1#2{\PY@reset\PY@toks#1+\relax+\PY@do{#2}}

\@namedef{PY@tok@w}{\def\PY@tc##1{\textcolor[rgb]{0.73,0.73,0.73}{##1}}}
\@namedef{PY@tok@c}{\let\PY@it=\textit\def\PY@tc##1{\textcolor[rgb]{0.24,0.48,0.48}{##1}}}
\@namedef{PY@tok@cp}{\def\PY@tc##1{\textcolor[rgb]{0.61,0.40,0.00}{##1}}}
\@namedef{PY@tok@k}{\let\PY@bf=\textbf\def\PY@tc##1{\textcolor[rgb]{0.00,0.50,0.00}{##1}}}
\@namedef{PY@tok@kp}{\def\PY@tc##1{\textcolor[rgb]{0.00,0.50,0.00}{##1}}}
\@namedef{PY@tok@kt}{\def\PY@tc##1{\textcolor[rgb]{0.69,0.00,0.25}{##1}}}
\@namedef{PY@tok@o}{\def\PY@tc##1{\textcolor[rgb]{0.40,0.40,0.40}{##1}}}
\@namedef{PY@tok@ow}{\let\PY@bf=\textbf\def\PY@tc##1{\textcolor[rgb]{0.67,0.13,1.00}{##1}}}
\@namedef{PY@tok@nb}{\def\PY@tc##1{\textcolor[rgb]{0.00,0.50,0.00}{##1}}}
\@namedef{PY@tok@nf}{\def\PY@tc##1{\textcolor[rgb]{0.00,0.00,1.00}{##1}}}
\@namedef{PY@tok@nc}{\let\PY@bf=\textbf\def\PY@tc##1{\textcolor[rgb]{0.00,0.00,1.00}{##1}}}
\@namedef{PY@tok@nn}{\let\PY@bf=\textbf\def\PY@tc##1{\textcolor[rgb]{0.00,0.00,1.00}{##1}}}
\@namedef{PY@tok@ne}{\let\PY@bf=\textbf\def\PY@tc##1{\textcolor[rgb]{0.80,0.25,0.22}{##1}}}
\@namedef{PY@tok@nv}{\def\PY@tc##1{\textcolor[rgb]{0.10,0.09,0.49}{##1}}}
\@namedef{PY@tok@no}{\def\PY@tc##1{\textcolor[rgb]{0.53,0.00,0.00}{##1}}}
\@namedef{PY@tok@nl}{\def\PY@tc##1{\textcolor[rgb]{0.46,0.46,0.00}{##1}}}
\@namedef{PY@tok@ni}{\let\PY@bf=\textbf\def\PY@tc##1{\textcolor[rgb]{0.44,0.44,0.44}{##1}}}
\@namedef{PY@tok@na}{\def\PY@tc##1{\textcolor[rgb]{0.41,0.47,0.13}{##1}}}
\@namedef{PY@tok@nt}{\let\PY@bf=\textbf\def\PY@tc##1{\textcolor[rgb]{0.00,0.50,0.00}{##1}}}
\@namedef{PY@tok@nd}{\def\PY@tc##1{\textcolor[rgb]{0.67,0.13,1.00}{##1}}}
\@namedef{PY@tok@s}{\def\PY@tc##1{\textcolor[rgb]{0.73,0.13,0.13}{##1}}}
\@namedef{PY@tok@sd}{\let\PY@it=\textit\def\PY@tc##1{\textcolor[rgb]{0.73,0.13,0.13}{##1}}}
\@namedef{PY@tok@si}{\let\PY@bf=\textbf\def\PY@tc##1{\textcolor[rgb]{0.64,0.35,0.47}{##1}}}
\@namedef{PY@tok@se}{\let\PY@bf=\textbf\def\PY@tc##1{\textcolor[rgb]{0.67,0.36,0.12}{##1}}}
\@namedef{PY@tok@sr}{\def\PY@tc##1{\textcolor[rgb]{0.64,0.35,0.47}{##1}}}
\@namedef{PY@tok@ss}{\def\PY@tc##1{\textcolor[rgb]{0.10,0.09,0.49}{##1}}}
\@namedef{PY@tok@sx}{\def\PY@tc##1{\textcolor[rgb]{0.00,0.50,0.00}{##1}}}
\@namedef{PY@tok@m}{\def\PY@tc##1{\textcolor[rgb]{0.40,0.40,0.40}{##1}}}
\@namedef{PY@tok@gh}{\let\PY@bf=\textbf\def\PY@tc##1{\textcolor[rgb]{0.00,0.00,0.50}{##1}}}
\@namedef{PY@tok@gu}{\let\PY@bf=\textbf\def\PY@tc##1{\textcolor[rgb]{0.50,0.00,0.50}{##1}}}
\@namedef{PY@tok@gd}{\def\PY@tc##1{\textcolor[rgb]{0.63,0.00,0.00}{##1}}}
\@namedef{PY@tok@gi}{\def\PY@tc##1{\textcolor[rgb]{0.00,0.52,0.00}{##1}}}
\@namedef{PY@tok@gr}{\def\PY@tc##1{\textcolor[rgb]{0.89,0.00,0.00}{##1}}}
\@namedef{PY@tok@ge}{\let\PY@it=\textit}
\@namedef{PY@tok@gs}{\let\PY@bf=\textbf}
\@namedef{PY@tok@gp}{\let\PY@bf=\textbf\def\PY@tc##1{\textcolor[rgb]{0.00,0.00,0.50}{##1}}}
\@namedef{PY@tok@go}{\def\PY@tc##1{\textcolor[rgb]{0.44,0.44,0.44}{##1}}}
\@namedef{PY@tok@gt}{\def\PY@tc##1{\textcolor[rgb]{0.00,0.27,0.87}{##1}}}
\@namedef{PY@tok@err}{\def\PY@bc##1{{\setlength{\fboxsep}{\string -\fboxrule}\fcolorbox[rgb]{1.00,0.00,0.00}{1,1,1}{\strut ##1}}}}
\@namedef{PY@tok@kc}{\let\PY@bf=\textbf\def\PY@tc##1{\textcolor[rgb]{0.00,0.50,0.00}{##1}}}
\@namedef{PY@tok@kd}{\let\PY@bf=\textbf\def\PY@tc##1{\textcolor[rgb]{0.00,0.50,0.00}{##1}}}
\@namedef{PY@tok@kn}{\let\PY@bf=\textbf\def\PY@tc##1{\textcolor[rgb]{0.00,0.50,0.00}{##1}}}
\@namedef{PY@tok@kr}{\let\PY@bf=\textbf\def\PY@tc##1{\textcolor[rgb]{0.00,0.50,0.00}{##1}}}
\@namedef{PY@tok@bp}{\def\PY@tc##1{\textcolor[rgb]{0.00,0.50,0.00}{##1}}}
\@namedef{PY@tok@fm}{\def\PY@tc##1{\textcolor[rgb]{0.00,0.00,1.00}{##1}}}
\@namedef{PY@tok@vc}{\def\PY@tc##1{\textcolor[rgb]{0.10,0.09,0.49}{##1}}}
\@namedef{PY@tok@vg}{\def\PY@tc##1{\textcolor[rgb]{0.10,0.09,0.49}{##1}}}
\@namedef{PY@tok@vi}{\def\PY@tc##1{\textcolor[rgb]{0.10,0.09,0.49}{##1}}}
\@namedef{PY@tok@vm}{\def\PY@tc##1{\textcolor[rgb]{0.10,0.09,0.49}{##1}}}
\@namedef{PY@tok@sa}{\def\PY@tc##1{\textcolor[rgb]{0.73,0.13,0.13}{##1}}}
\@namedef{PY@tok@sb}{\def\PY@tc##1{\textcolor[rgb]{0.73,0.13,0.13}{##1}}}
\@namedef{PY@tok@sc}{\def\PY@tc##1{\textcolor[rgb]{0.73,0.13,0.13}{##1}}}
\@namedef{PY@tok@dl}{\def\PY@tc##1{\textcolor[rgb]{0.73,0.13,0.13}{##1}}}
\@namedef{PY@tok@s2}{\def\PY@tc##1{\textcolor[rgb]{0.73,0.13,0.13}{##1}}}
\@namedef{PY@tok@sh}{\def\PY@tc##1{\textcolor[rgb]{0.73,0.13,0.13}{##1}}}
\@namedef{PY@tok@s1}{\def\PY@tc##1{\textcolor[rgb]{0.73,0.13,0.13}{##1}}}
\@namedef{PY@tok@mb}{\def\PY@tc##1{\textcolor[rgb]{0.40,0.40,0.40}{##1}}}
\@namedef{PY@tok@mf}{\def\PY@tc##1{\textcolor[rgb]{0.40,0.40,0.40}{##1}}}
\@namedef{PY@tok@mh}{\def\PY@tc##1{\textcolor[rgb]{0.40,0.40,0.40}{##1}}}
\@namedef{PY@tok@mi}{\def\PY@tc##1{\textcolor[rgb]{0.40,0.40,0.40}{##1}}}
\@namedef{PY@tok@il}{\def\PY@tc##1{\textcolor[rgb]{0.40,0.40,0.40}{##1}}}
\@namedef{PY@tok@mo}{\def\PY@tc##1{\textcolor[rgb]{0.40,0.40,0.40}{##1}}}
\@namedef{PY@tok@ch}{\let\PY@it=\textit\def\PY@tc##1{\textcolor[rgb]{0.24,0.48,0.48}{##1}}}
\@namedef{PY@tok@cm}{\let\PY@it=\textit\def\PY@tc##1{\textcolor[rgb]{0.24,0.48,0.48}{##1}}}
\@namedef{PY@tok@cpf}{\let\PY@it=\textit\def\PY@tc##1{\textcolor[rgb]{0.24,0.48,0.48}{##1}}}
\@namedef{PY@tok@c1}{\let\PY@it=\textit\def\PY@tc##1{\textcolor[rgb]{0.24,0.48,0.48}{##1}}}
\@namedef{PY@tok@cs}{\let\PY@it=\textit\def\PY@tc##1{\textcolor[rgb]{0.24,0.48,0.48}{##1}}}

\def\PYZbs{\char`\\}
\def\PYZus{\char`\_}
\def\PYZob{\char`\{}
\def\PYZcb{\char`\}}
\def\PYZca{\char`\^}
\def\PYZam{\char`\&}
\def\PYZlt{\char`\<}
\def\PYZgt{\char`\>}
\def\PYZsh{\char`\#}
\def\PYZpc{\char`\%}
\def\PYZdl{\char`\$}
\def\PYZhy{\char`\-}
\def\PYZsq{\char`\'}
\def\PYZdq{\char`\"}
\def\PYZti{\char`\~}
% for compatibility with earlier versions
\def\PYZat{@}
\def\PYZlb{[}
\def\PYZrb{]}
\makeatother


    % For linebreaks inside Verbatim environment from package fancyvrb.
    \makeatletter
        \newbox\Wrappedcontinuationbox
        \newbox\Wrappedvisiblespacebox
        \newcommand*\Wrappedvisiblespace {\textcolor{red}{\textvisiblespace}}
        \newcommand*\Wrappedcontinuationsymbol {\textcolor{red}{\llap{\tiny$\m@th\hookrightarrow$}}}
        \newcommand*\Wrappedcontinuationindent {3ex }
        \newcommand*\Wrappedafterbreak {\kern\Wrappedcontinuationindent\copy\Wrappedcontinuationbox}
        % Take advantage of the already applied Pygments mark-up to insert
        % potential linebreaks for TeX processing.
        %        {, <, #, %, $, ' and ": go to next line.
        %        _, }, ^, &, >, - and ~: stay at end of broken line.
        % Use of \textquotesingle for straight quote.
        \newcommand*\Wrappedbreaksatspecials {%
            \def\PYGZus{\discretionary{\char`\_}{\Wrappedafterbreak}{\char`\_}}%
            \def\PYGZob{\discretionary{}{\Wrappedafterbreak\char`\{}{\char`\{}}%
            \def\PYGZcb{\discretionary{\char`\}}{\Wrappedafterbreak}{\char`\}}}%
            \def\PYGZca{\discretionary{\char`\^}{\Wrappedafterbreak}{\char`\^}}%
            \def\PYGZam{\discretionary{\char`\&}{\Wrappedafterbreak}{\char`\&}}%
            \def\PYGZlt{\discretionary{}{\Wrappedafterbreak\char`\<}{\char`\<}}%
            \def\PYGZgt{\discretionary{\char`\>}{\Wrappedafterbreak}{\char`\>}}%
            \def\PYGZsh{\discretionary{}{\Wrappedafterbreak\char`\#}{\char`\#}}%
            \def\PYGZpc{\discretionary{}{\Wrappedafterbreak\char`\%}{\char`\%}}%
            \def\PYGZdl{\discretionary{}{\Wrappedafterbreak\char`\$}{\char`\$}}%
            \def\PYGZhy{\discretionary{\char`\-}{\Wrappedafterbreak}{\char`\-}}%
            \def\PYGZsq{\discretionary{}{\Wrappedafterbreak\textquotesingle}{\textquotesingle}}%
            \def\PYGZdq{\discretionary{}{\Wrappedafterbreak\char`\"}{\char`\"}}%
            \def\PYGZti{\discretionary{\char`\~}{\Wrappedafterbreak}{\char`\~}}%
        }
        % Some characters . , ; ? ! / are not pygmentized.
        % This macro makes them "active" and they will insert potential linebreaks
        \newcommand*\Wrappedbreaksatpunct {%
            \lccode`\~`\.\lowercase{\def~}{\discretionary{\hbox{\char`\.}}{\Wrappedafterbreak}{\hbox{\char`\.}}}%
            \lccode`\~`\,\lowercase{\def~}{\discretionary{\hbox{\char`\,}}{\Wrappedafterbreak}{\hbox{\char`\,}}}%
            \lccode`\~`\;\lowercase{\def~}{\discretionary{\hbox{\char`\;}}{\Wrappedafterbreak}{\hbox{\char`\;}}}%
            \lccode`\~`\:\lowercase{\def~}{\discretionary{\hbox{\char`\:}}{\Wrappedafterbreak}{\hbox{\char`\:}}}%
            \lccode`\~`\?\lowercase{\def~}{\discretionary{\hbox{\char`\?}}{\Wrappedafterbreak}{\hbox{\char`\?}}}%
            \lccode`\~`\!\lowercase{\def~}{\discretionary{\hbox{\char`\!}}{\Wrappedafterbreak}{\hbox{\char`\!}}}%
            \lccode`\~`\/\lowercase{\def~}{\discretionary{\hbox{\char`\/}}{\Wrappedafterbreak}{\hbox{\char`\/}}}%
            \catcode`\.\active
            \catcode`\,\active
            \catcode`\;\active
            \catcode`\:\active
            \catcode`\?\active
            \catcode`\!\active
            \catcode`\/\active
            \lccode`\~`\~
        }
    \makeatother

    \let\OriginalVerbatim=\Verbatim
    \makeatletter
    \renewcommand{\Verbatim}[1][1]{%
        %\parskip\z@skip
        \sbox\Wrappedcontinuationbox {\Wrappedcontinuationsymbol}%
        \sbox\Wrappedvisiblespacebox {\FV@SetupFont\Wrappedvisiblespace}%
        \def\FancyVerbFormatLine ##1{\hsize\linewidth
            \vtop{\raggedright\hyphenpenalty\z@\exhyphenpenalty\z@
                \doublehyphendemerits\z@\finalhyphendemerits\z@
                \strut ##1\strut}%
        }%
        % If the linebreak is at a space, the latter will be displayed as visible
        % space at end of first line, and a continuation symbol starts next line.
        % Stretch/shrink are however usually zero for typewriter font.
        \def\FV@Space {%
            \nobreak\hskip\z@ plus\fontdimen3\font minus\fontdimen4\font
            \discretionary{\copy\Wrappedvisiblespacebox}{\Wrappedafterbreak}
            {\kern\fontdimen2\font}%
        }%

        % Allow breaks at special characters using \PYG... macros.
        \Wrappedbreaksatspecials
        % Breaks at punctuation characters . , ; ? ! and / need catcode=\active
        \OriginalVerbatim[#1,codes*=\Wrappedbreaksatpunct]%
    }
    \makeatother

    % Exact colors from NB
    \definecolor{incolor}{HTML}{303F9F}
    \definecolor{outcolor}{HTML}{D84315}
    \definecolor{cellborder}{HTML}{CFCFCF}
    \definecolor{cellbackground}{HTML}{F7F7F7}

    % prompt
    \makeatletter
    \newcommand{\boxspacing}{\kern\kvtcb@left@rule\kern\kvtcb@boxsep}
    \makeatother
    \newcommand{\prompt}[4]{
        {\ttfamily\llap{{\color{#2}[#3]:\hspace{3pt}#4}}\vspace{-\baselineskip}}
    }
    

    
    % Prevent overflowing lines due to hard-to-break entities
    \sloppy
    % Setup hyperref package
    \hypersetup{
      breaklinks=true,  % so long urls are correctly broken across lines
      colorlinks=true,
      urlcolor=urlcolor,
      linkcolor=linkcolor,
      citecolor=citecolor,
      }
    % Slightly bigger margins than the latex defaults
    
    \geometry{verbose,tmargin=1in,bmargin=1in,lmargin=1in,rmargin=1in}
    
    

\begin{document}
    
    \maketitle
    
    

    
    \section{Introducción}\label{introducciuxf3n}

\paragraph{En el siguiente cuaderno se presenta la implementacion de una
red neuronal capaz de identificar numeros manuscritos, para el
desarrollo fueron utilizados los datos correspondientes al dataset Mnist
alojado en keras. Inicialmente se realizó un procesado de las imagenes y
de sus respectivas etiquetas de forma que podamos contribuir al modelo
con datos que faciliten y optimicen sus procesos de
aprendizaje.}\label{en-el-siguiente-cuaderno-se-presenta-la-implementacion-de-una-red-neuronal-capaz-de-identificar-numeros-manuscritos-para-el-desarrollo-fueron-utilizados-los-datos-correspondientes-al-dataset-mnist-alojado-en-keras.-inicialmente-se-realizuxf3-un-procesado-de-las-imagenes-y-de-sus-respectivas-etiquetas-de-forma-que-podamos-contribuir-al-modelo-con-datos-que-faciliten-y-optimicen-sus-procesos-de-aprendizaje.}

\paragraph{Una vez definido el procesamiento de los datos, la forma de
las entradas y las salidas; se definió el modelo de red neuronal que se
utilizaria, dicho modelo cuenta con 6 capas con numeros de neuronas
variables entre las 256 y las 16, el entrenamiento de la red neuronal se
realizo en lotes(batches) lo cual mostró un comportamiento particular en
el proceso de
entrenamiento.}\label{una-vez-definido-el-procesamiento-de-los-datos-la-forma-de-las-entradas-y-las-salidas-se-definiuxf3-el-modelo-de-red-neuronal-que-se-utilizaria-dicho-modelo-cuenta-con-6-capas-con-numeros-de-neuronas-variables-entre-las-256-y-las-16-el-entrenamiento-de-la-red-neuronal-se-realizo-en-lotesbatches-lo-cual-mostruxf3-un-comportamiento-particular-en-el-proceso-de-entrenamiento.}

\paragraph{Finalmente, luego de varias los parametros, la cantidad de
iteraciones, cambiar el dropout, se obtuvo una precision del 92\% sin
embargo al momento de realizar las pruebas se identificó que el modelo
parece presentar un sesgo pues muchas imagenes que intento predecir
tomaron este valor aparentemente arbitrario ante problemas no conocidos.
Con el fin de mejorar los resultados se implementa un segundo modelo
esta vez haciendo uso de procesos de convolucion y/o funciones de
activacion no
lineales}\label{finalmente-luego-de-varias-los-parametros-la-cantidad-de-iteraciones-cambiar-el-dropout-se-obtuvo-una-precision-del-92-sin-embargo-al-momento-de-realizar-las-pruebas-se-identificuxf3-que-el-modelo-parece-presentar-un-sesgo-pues-muchas-imagenes-que-intento-predecir-tomaron-este-valor-aparentemente-arbitrario-ante-problemas-no-conocidos.-con-el-fin-de-mejorar-los-resultados-se-implementa-un-segundo-modelo-esta-vez-haciendo-uso-de-procesos-de-convolucion-yo-funciones-de-activacion-no-lineales}

    \begin{tcolorbox}[breakable, size=fbox, boxrule=1pt, pad at break*=1mm,colback=cellbackground, colframe=cellborder]
\prompt{In}{incolor}{2}{\boxspacing}
\begin{Verbatim}[commandchars=\\\{\}]
\PY{k+kn}{import} \PY{n+nn}{tensorflow} \PY{k}{as} \PY{n+nn}{tf}
\PY{k+kn}{import} \PY{n+nn}{numpy} \PY{k}{as} \PY{n+nn}{np}
\PY{k+kn}{from} \PY{n+nn}{PIL} \PY{k+kn}{import} \PY{n}{Image}\PY{p}{,} \PY{n}{ImageOps}
\PY{k+kn}{import} \PY{n+nn}{os}
\PY{k+kn}{import} \PY{n+nn}{matplotlib}\PY{n+nn}{.}\PY{n+nn}{pyplot} \PY{k}{as} \PY{n+nn}{plt}
\PY{k+kn}{import} \PY{n+nn}{math}
\PY{k+kn}{import} \PY{n+nn}{random}

\PY{n}{tf}\PY{o}{.}\PY{n}{compat}\PY{o}{.}\PY{n}{v1}\PY{o}{.}\PY{n}{disable\PYZus{}eager\PYZus{}execution}\PY{p}{(}\PY{p}{)}
\end{Verbatim}
\end{tcolorbox}

    \begin{tcolorbox}[breakable, size=fbox, boxrule=1pt, pad at break*=1mm,colback=cellbackground, colframe=cellborder]
\prompt{In}{incolor}{3}{\boxspacing}
\begin{Verbatim}[commandchars=\\\{\}]
\PY{k+kn}{from} \PY{n+nn}{tensorflow} \PY{k+kn}{import} \PY{n}{keras}

\PY{n}{mnist} \PY{o}{=} \PY{n}{keras}\PY{o}{.}\PY{n}{datasets}\PY{o}{.}\PY{n}{mnist}

\PY{p}{(}\PY{n}{train\PYZus{}images}\PY{p}{,} \PY{n}{train\PYZus{}labels}\PY{p}{)}\PY{p}{,}\PY{p}{(}\PY{n}{test\PYZus{}images}\PY{p}{,} \PY{n}{test\PYZus{}labels}\PY{p}{)} \PY{o}{=} \PY{n}{mnist}\PY{o}{.}\PY{n}{load\PYZus{}data}\PY{p}{(}\PY{p}{)}
\end{Verbatim}
\end{tcolorbox}

    \paragraph{Para realizar las pruebas tomaremos 10 imagenes del dataset
para verificar la precision del mismo observando las imagenes a evaluar
estas se almacenaran en el arreglo val\_images y
val\_labels}\label{para-realizar-las-pruebas-tomaremos-10-imagenes-del-dataset-para-verificar-la-precision-del-mismo-observando-las-imagenes-a-evaluar-estas-se-almacenaran-en-el-arreglo-val_images-y-val_labels}

    \begin{tcolorbox}[breakable, size=fbox, boxrule=1pt, pad at break*=1mm,colback=cellbackground, colframe=cellborder]
\prompt{In}{incolor}{4}{\boxspacing}
\begin{Verbatim}[commandchars=\\\{\}]
\PY{n}{val\PYZus{}images} \PY{o}{=} \PY{n}{test\PYZus{}images}\PY{p}{[}\PY{o}{\PYZhy{}}\PY{l+m+mi}{10}\PY{p}{:}\PY{p}{,}\PY{p}{:}\PY{p}{,}\PY{p}{:}\PY{p}{]}
\PY{n}{val\PYZus{}labels} \PY{o}{=} \PY{n}{test\PYZus{}labels}\PY{p}{[}\PY{o}{\PYZhy{}}\PY{l+m+mi}{10}\PY{p}{:}\PY{p}{]}
\PY{n}{test\PYZus{}images} \PY{o}{=} \PY{n}{test\PYZus{}images}\PY{p}{[}\PY{p}{:}\PY{o}{\PYZhy{}}\PY{l+m+mi}{10}\PY{p}{,}\PY{p}{:}\PY{p}{,}\PY{p}{:}\PY{p}{]}
\PY{n}{test\PYZus{}labels} \PY{o}{=} \PY{n}{test\PYZus{}labels}\PY{p}{[}\PY{p}{:}\PY{o}{\PYZhy{}}\PY{l+m+mi}{10}\PY{p}{]}
\end{Verbatim}
\end{tcolorbox}

    \subsubsection{Visualizando algunas imagenes del conjunto
disponible}\label{visualizando-algunas-imagenes-del-conjunto-disponible}

    \begin{tcolorbox}[breakable, size=fbox, boxrule=1pt, pad at break*=1mm,colback=cellbackground, colframe=cellborder]
\prompt{In}{incolor}{5}{\boxspacing}
\begin{Verbatim}[commandchars=\\\{\}]
\PY{n}{plt}\PY{o}{.}\PY{n}{figure}\PY{p}{(}\PY{n}{figsize}\PY{o}{=}\PY{p}{(}\PY{l+m+mi}{5}\PY{p}{,}\PY{l+m+mi}{5}\PY{p}{)}\PY{p}{)}
\PY{k}{for} \PY{n}{i} \PY{o+ow}{in} \PY{n+nb}{range}\PY{p}{(}\PY{l+m+mi}{25}\PY{p}{)}\PY{p}{:}
    \PY{n}{plt}\PY{o}{.}\PY{n}{subplot}\PY{p}{(}\PY{l+m+mi}{5}\PY{p}{,}\PY{l+m+mi}{5}\PY{p}{,}\PY{n}{i}\PY{o}{+}\PY{l+m+mi}{1}\PY{p}{)}
    \PY{n}{plt}\PY{o}{.}\PY{n}{xticks}\PY{p}{(}\PY{p}{[}\PY{p}{]}\PY{p}{)}
    \PY{n}{plt}\PY{o}{.}\PY{n}{yticks}\PY{p}{(}\PY{p}{[}\PY{p}{]}\PY{p}{)}
    \PY{n}{plt}\PY{o}{.}\PY{n}{grid}\PY{p}{(}\PY{k+kc}{False}\PY{p}{)}
    \PY{n}{k} \PY{o}{=} \PY{n}{random}\PY{o}{.}\PY{n}{randint}\PY{p}{(}\PY{l+m+mi}{0}\PY{p}{,} \PY{n+nb}{len}\PY{p}{(}\PY{n}{train\PYZus{}images}\PY{p}{)}\PY{p}{)}
    \PY{n}{plt}\PY{o}{.}\PY{n}{imshow}\PY{p}{(}\PY{n}{train\PYZus{}images}\PY{p}{[}\PY{n}{k}\PY{p}{]}\PY{p}{,} \PY{n}{cmap}\PY{o}{=}\PY{n}{plt}\PY{o}{.}\PY{n}{cm}\PY{o}{.}\PY{n}{gray}\PY{p}{)}
    \PY{n}{plt}\PY{o}{.}\PY{n}{xlabel}\PY{p}{(}\PY{n}{train\PYZus{}labels}\PY{p}{[}\PY{n}{k}\PY{p}{]}\PY{p}{)}
\PY{n}{plt}\PY{o}{.}\PY{n}{show}\PY{p}{(}\PY{p}{)}
\end{Verbatim}
\end{tcolorbox}

    \begin{center}
    \adjustimage{max size={0.9\linewidth}{0.9\paperheight}}{output_6_0.png}
    \end{center}
    { \hspace*{\fill} \\}
    
    \section{Depuración / Pre-Procesamiento de los
datos}\label{depuraciuxf3n-pre-procesamiento-de-los-datos}

    \paragraph{Se realiza la division de los datos con el fin de que solo
ocupen valores 0 o 1, en otras palabras imagenes en blanco y
negro}\label{se-realiza-la-division-de-los-datos-con-el-fin-de-que-solo-ocupen-valores-0-o-1-en-otras-palabras-imagenes-en-blanco-y-negro}

    \begin{tcolorbox}[breakable, size=fbox, boxrule=1pt, pad at break*=1mm,colback=cellbackground, colframe=cellborder]
\prompt{In}{incolor}{6}{\boxspacing}
\begin{Verbatim}[commandchars=\\\{\}]
\PY{n}{train\PYZus{}images} \PY{o}{=} \PY{n}{train\PYZus{}images} \PY{o}{/} \PY{l+m+mf}{255.0}
\PY{n}{test\PYZus{}images} \PY{o}{=} \PY{n}{test\PYZus{}images} \PY{o}{/} \PY{l+m+mf}{255.0}
\end{Verbatim}
\end{tcolorbox}

    \paragraph{Con el animo de contribuir con el modelo es necesario
normalizar las salidas o etiquetas, en este caso son numeros entre 0y9
por lo que podrian interpretarse como 10 categorias las cuales se
plasman en un arreglo de 10 posiciones donde a cada categoria(digito) le
corresponde un arreglo con todas las posiciones en 0 salvo la posicion
correspondiente}\label{con-el-animo-de-contribuir-con-el-modelo-es-necesario-normalizar-las-salidas-o-etiquetas-en-este-caso-son-numeros-entre-0y9-por-lo-que-podrian-interpretarse-como-10-categorias-las-cuales-se-plasman-en-un-arreglo-de-10-posiciones-donde-a-cada-categoriadigito-le-corresponde-un-arreglo-con-todas-las-posiciones-en-0-salvo-la-posicion-correspondiente}

    \begin{tcolorbox}[breakable, size=fbox, boxrule=1pt, pad at break*=1mm,colback=cellbackground, colframe=cellborder]
\prompt{In}{incolor}{7}{\boxspacing}
\begin{Verbatim}[commandchars=\\\{\}]
\PY{k}{def} \PY{n+nf}{oneHotTarget}\PY{p}{(}\PY{n}{targetColumn}\PY{p}{)}\PY{p}{:}
    \PY{n}{train\PYZus{}labels\PYZus{}aux} \PY{o}{=} \PY{n}{np}\PY{o}{.}\PY{n}{array}\PY{p}{(}\PY{n}{targetColumn}\PY{p}{)}\PY{o}{.}\PY{n}{astype}\PY{p}{(}\PY{n+nb}{int}\PY{p}{)}
    \PY{n}{labels\PYZus{}OH} \PY{o}{=} \PY{n}{np}\PY{o}{.}\PY{n}{zeros}\PY{p}{(}\PY{p}{(}\PY{n}{train\PYZus{}labels\PYZus{}aux}\PY{o}{.}\PY{n}{size}\PY{p}{,} \PY{n}{train\PYZus{}labels\PYZus{}aux}\PY{o}{.}\PY{n}{max}\PY{p}{(}\PY{p}{)}\PY{o}{+}\PY{l+m+mi}{1}\PY{p}{)}\PY{p}{)}
    \PY{n}{labels\PYZus{}OH}\PY{p}{[}\PY{n}{np}\PY{o}{.}\PY{n}{arange}\PY{p}{(}\PY{n}{train\PYZus{}labels\PYZus{}aux}\PY{o}{.}\PY{n}{size}\PY{p}{)}\PY{p}{,} \PY{n}{train\PYZus{}labels\PYZus{}aux}\PY{p}{]} \PY{o}{=} \PY{l+m+mi}{1}
    \PY{k}{return} \PY{n}{labels\PYZus{}OH}
    



\PY{n}{train\PYZus{}labels\PYZus{}OH} \PY{o}{=} \PY{n}{oneHotTarget}\PY{p}{(}\PY{n}{train\PYZus{}labels}\PY{p}{)}
\PY{n}{test\PYZus{}labels\PYZus{}OH} \PY{o}{=} \PY{n}{oneHotTarget}\PY{p}{(}\PY{n}{test\PYZus{}labels}\PY{p}{)}



\PY{c+c1}{\PYZsh{}\PYZsh{}\PYZsh{}\PYZsh{}\PYZsh{} Es necesario convertir las columnas de labels en one\PYZhy{}hot}
\PY{n+nb}{print}\PY{p}{(}\PY{n}{train\PYZus{}labels}\PY{p}{[}\PY{l+m+mi}{0}\PY{p}{]}\PY{p}{)}
\PY{n+nb}{print}\PY{p}{(}\PY{n}{train\PYZus{}labels\PYZus{}OH}\PY{p}{[}\PY{l+m+mi}{0}\PY{p}{]}\PY{p}{)}
\end{Verbatim}
\end{tcolorbox}

    \begin{Verbatim}[commandchars=\\\{\}]
5
[0. 0. 0. 0. 0. 1. 0. 0. 0. 0.]
    \end{Verbatim}

    \section{Definición / implementación del
modelo}\label{definiciuxf3n-implementaciuxf3n-del-modelo}

    \begin{tcolorbox}[breakable, size=fbox, boxrule=1pt, pad at break*=1mm,colback=cellbackground, colframe=cellborder]
\prompt{In}{incolor}{8}{\boxspacing}
\begin{Verbatim}[commandchars=\\\{\}]
\PY{n+nb}{print}\PY{p}{(}\PY{n}{train\PYZus{}images}\PY{p}{[}\PY{l+m+mi}{0}\PY{p}{]}\PY{o}{.}\PY{n}{shape}\PY{p}{)}
\end{Verbatim}
\end{tcolorbox}

    \begin{Verbatim}[commandchars=\\\{\}]
(28, 28)
    \end{Verbatim}

    \paragraph{Verificamos las dimensiones de las imagenes y las
multiplicamos para conocer el tamaño de las entradas, es importante
tener en cuenta que para este modelo es necesario convertir las entradas
en arreglos unidimensionales del tamaño obtenido por la multiplicacion
de sus filas y columnas. Tambien se define el tamaño de la salida que de
acuerdo con el proceso realizado en el paso anterior corresponde a 10
por la codificacion de las
etiquetas.}\label{verificamos-las-dimensiones-de-las-imagenes-y-las-multiplicamos-para-conocer-el-tamauxf1o-de-las-entradas-es-importante-tener-en-cuenta-que-para-este-modelo-es-necesario-convertir-las-entradas-en-arreglos-unidimensionales-del-tamauxf1o-obtenido-por-la-multiplicacion-de-sus-filas-y-columnas.-tambien-se-define-el-tamauxf1o-de-la-salida-que-de-acuerdo-con-el-proceso-realizado-en-el-paso-anterior-corresponde-a-10-por-la-codificacion-de-las-etiquetas.}

\paragraph{Adicionalmente se crea el tensor correspondiente al control
del rango de
dropout.}\label{adicionalmente-se-crea-el-tensor-correspondiente-al-control-del-rango-de-dropout.}

    \begin{tcolorbox}[breakable, size=fbox, boxrule=1pt, pad at break*=1mm,colback=cellbackground, colframe=cellborder]
\prompt{In}{incolor}{44}{\boxspacing}
\begin{Verbatim}[commandchars=\\\{\}]
\PY{n}{plt}\PY{o}{.}\PY{n}{figure}\PY{p}{(}\PY{n}{figsize}\PY{o}{=}\PY{p}{(}\PY{l+m+mi}{9}\PY{p}{,}\PY{l+m+mi}{10}\PY{p}{)}\PY{p}{)}
\PY{n}{rnn} \PY{o}{=} \PY{n}{Image}\PY{o}{.}\PY{n}{open}\PY{p}{(}\PY{l+s+s1}{\PYZsq{}}\PY{l+s+s1}{modelo ann.png}\PY{l+s+s1}{\PYZsq{}}\PY{p}{)}
\PY{n}{plt}\PY{o}{.}\PY{n}{xticks}\PY{p}{(}\PY{p}{[}\PY{p}{]}\PY{p}{)}
\PY{n}{plt}\PY{o}{.}\PY{n}{yticks}\PY{p}{(}\PY{p}{[}\PY{p}{]}\PY{p}{)}
\PY{n}{plt}\PY{o}{.}\PY{n}{grid}\PY{p}{(}\PY{k+kc}{False}\PY{p}{)}
\PY{n}{plt}\PY{o}{.}\PY{n}{imshow}\PY{p}{(}\PY{n}{rnn}\PY{p}{)}
\PY{n}{plt}\PY{o}{.}\PY{n}{show}\PY{p}{(}\PY{p}{)}
\end{Verbatim}
\end{tcolorbox}

    \begin{center}
    \adjustimage{max size={0.9\linewidth}{0.9\paperheight}}{output_15_0.png}
    \end{center}
    { \hspace*{\fill} \\}
    
    \begin{tcolorbox}[breakable, size=fbox, boxrule=1pt, pad at break*=1mm,colback=cellbackground, colframe=cellborder]
\prompt{In}{incolor}{9}{\boxspacing}
\begin{Verbatim}[commandchars=\\\{\}]
\PY{n}{X} \PY{o}{=} \PY{n}{tf}\PY{o}{.}\PY{n}{compat}\PY{o}{.}\PY{n}{v1}\PY{o}{.}\PY{n}{placeholder}\PY{p}{(}\PY{l+s+s2}{\PYZdq{}}\PY{l+s+s2}{float}\PY{l+s+s2}{\PYZdq{}}\PY{p}{,}\PY{p}{[}\PY{k+kc}{None}\PY{p}{,} \PY{l+m+mi}{784}\PY{p}{]}\PY{p}{)} \PY{c+c1}{\PYZsh{}\PYZsh{}\PYZsh{} 28x28}
\PY{n}{Y} \PY{o}{=} \PY{n}{tf}\PY{o}{.}\PY{n}{compat}\PY{o}{.}\PY{n}{v1}\PY{o}{.}\PY{n}{placeholder}\PY{p}{(}\PY{l+s+s2}{\PYZdq{}}\PY{l+s+s2}{float}\PY{l+s+s2}{\PYZdq{}}\PY{p}{,}\PY{p}{[}\PY{k+kc}{None}\PY{p}{,} \PY{l+m+mi}{10}\PY{p}{]}\PY{p}{)} \PY{c+c1}{\PYZsh{}\PYZsh{}\PYZsh{} los parametros indican el tipo de datos que contendra el tensor}
                                                 \PY{c+c1}{\PYZsh{}\PYZsh{}\PYZsh{} y las dimensiones}
                                                 \PY{c+c1}{\PYZsh{}\PYZsh{}\PYZsh{} None significa ilimitado. [numero de entradas, de tamaño]}
\PY{n}{keep\PYZus{}prob} \PY{o}{=} \PY{n}{tf}\PY{o}{.}\PY{n}{compat}\PY{o}{.}\PY{n}{v1}\PY{o}{.}\PY{n}{placeholder}\PY{p}{(}\PY{n}{tf}\PY{o}{.}\PY{n}{float32}\PY{p}{)}
\end{Verbatim}
\end{tcolorbox}

    \paragraph{Se definieron 2 diccionarios de tensores correspondientes a
los pesos y los bias que son los valores a calcular y refinar en cada
iteracion que realice el modelo en el proceso de entrenamiento, notamos
que cada uno de los tensores tiene medidas diferente, esto hace
referencia a la cantidad de neuronas que sera utilizada en cada una de
las capas definidas en las siguientes
celdas}\label{se-definieron-2-diccionarios-de-tensores-correspondientes-a-los-pesos-y-los-bias-que-son-los-valores-a-calcular-y-refinar-en-cada-iteracion-que-realice-el-modelo-en-el-proceso-de-entrenamiento-notamos-que-cada-uno-de-los-tensores-tiene-medidas-diferente-esto-hace-referencia-a-la-cantidad-de-neuronas-que-sera-utilizada-en-cada-una-de-las-capas-definidas-en-las-siguientes-celdas}

    \begin{tcolorbox}[breakable, size=fbox, boxrule=1pt, pad at break*=1mm,colback=cellbackground, colframe=cellborder]
\prompt{In}{incolor}{10}{\boxspacing}
\begin{Verbatim}[commandchars=\\\{\}]
\PY{n}{wlst}\PY{o}{=}\PY{p}{[}\PY{l+s+s1}{\PYZsq{}}\PY{l+s+s1}{w1}\PY{l+s+s1}{\PYZsq{}}\PY{p}{,}\PY{l+s+s1}{\PYZsq{}}\PY{l+s+s1}{w2}\PY{l+s+s1}{\PYZsq{}}\PY{p}{,}\PY{l+s+s1}{\PYZsq{}}\PY{l+s+s1}{w3}\PY{l+s+s1}{\PYZsq{}}\PY{p}{,}\PY{l+s+s1}{\PYZsq{}}\PY{l+s+s1}{w4}\PY{l+s+s1}{\PYZsq{}}\PY{p}{,}\PY{l+s+s1}{\PYZsq{}}\PY{l+s+s1}{w5}\PY{l+s+s1}{\PYZsq{}}\PY{p}{,}\PY{l+s+s1}{\PYZsq{}}\PY{l+s+s1}{w6}\PY{l+s+s1}{\PYZsq{}}\PY{p}{,}\PY{l+s+s1}{\PYZsq{}}\PY{l+s+s1}{out}\PY{l+s+s1}{\PYZsq{}}\PY{p}{]}
\PY{n}{biaslst}\PY{o}{=}\PY{p}{[}\PY{l+s+s1}{\PYZsq{}}\PY{l+s+s1}{b1}\PY{l+s+s1}{\PYZsq{}}\PY{p}{,}\PY{l+s+s1}{\PYZsq{}}\PY{l+s+s1}{b2}\PY{l+s+s1}{\PYZsq{}}\PY{p}{,}\PY{l+s+s1}{\PYZsq{}}\PY{l+s+s1}{b3}\PY{l+s+s1}{\PYZsq{}}\PY{p}{,}\PY{l+s+s1}{\PYZsq{}}\PY{l+s+s1}{b4}\PY{l+s+s1}{\PYZsq{}}\PY{p}{,}\PY{l+s+s1}{\PYZsq{}}\PY{l+s+s1}{b5}\PY{l+s+s1}{\PYZsq{}}\PY{p}{,}\PY{l+s+s1}{\PYZsq{}}\PY{l+s+s1}{b6}\PY{l+s+s1}{\PYZsq{}}\PY{p}{,}\PY{l+s+s1}{\PYZsq{}}\PY{l+s+s1}{out}\PY{l+s+s1}{\PYZsq{}}\PY{p}{]}

\PY{n}{w} \PY{o}{=} \PY{p}{\PYZob{}}
    \PY{l+s+s1}{\PYZsq{}}\PY{l+s+s1}{w1}\PY{l+s+s1}{\PYZsq{}}\PY{p}{:}\PY{n}{tf}\PY{o}{.}\PY{n}{Variable}\PY{p}{(}\PY{n}{tf}\PY{o}{.}\PY{n}{random}\PY{o}{.}\PY{n}{truncated\PYZus{}normal}\PY{p}{(}\PY{p}{[}\PY{l+m+mi}{784}\PY{p}{,}\PY{l+m+mi}{256}\PY{p}{]}\PY{p}{,} \PY{n}{stddev}\PY{o}{=}\PY{l+m+mf}{0.1}\PY{p}{)}\PY{p}{)}\PY{p}{,}
    \PY{l+s+s1}{\PYZsq{}}\PY{l+s+s1}{w2}\PY{l+s+s1}{\PYZsq{}}\PY{p}{:}\PY{n}{tf}\PY{o}{.}\PY{n}{Variable}\PY{p}{(}\PY{n}{tf}\PY{o}{.}\PY{n}{random}\PY{o}{.}\PY{n}{truncated\PYZus{}normal}\PY{p}{(}\PY{p}{[}\PY{l+m+mi}{256}\PY{p}{,}\PY{l+m+mi}{256}\PY{p}{]}\PY{p}{,} \PY{n}{stddev}\PY{o}{=}\PY{l+m+mf}{0.1}\PY{p}{)}\PY{p}{)}\PY{p}{,}
    \PY{l+s+s1}{\PYZsq{}}\PY{l+s+s1}{w3}\PY{l+s+s1}{\PYZsq{}}\PY{p}{:}\PY{n}{tf}\PY{o}{.}\PY{n}{Variable}\PY{p}{(}\PY{n}{tf}\PY{o}{.}\PY{n}{random}\PY{o}{.}\PY{n}{truncated\PYZus{}normal}\PY{p}{(}\PY{p}{[}\PY{l+m+mi}{256}\PY{p}{,}\PY{l+m+mi}{128}\PY{p}{]}\PY{p}{,} \PY{n}{stddev}\PY{o}{=}\PY{l+m+mf}{0.1}\PY{p}{)}\PY{p}{)}\PY{p}{,}
    \PY{l+s+s1}{\PYZsq{}}\PY{l+s+s1}{w4}\PY{l+s+s1}{\PYZsq{}}\PY{p}{:}\PY{n}{tf}\PY{o}{.}\PY{n}{Variable}\PY{p}{(}\PY{n}{tf}\PY{o}{.}\PY{n}{random}\PY{o}{.}\PY{n}{truncated\PYZus{}normal}\PY{p}{(}\PY{p}{[}\PY{l+m+mi}{128}\PY{p}{,}\PY{l+m+mi}{64}\PY{p}{]}\PY{p}{,} \PY{n}{stddev}\PY{o}{=}\PY{l+m+mf}{0.1}\PY{p}{)}\PY{p}{)}\PY{p}{,}
    \PY{l+s+s1}{\PYZsq{}}\PY{l+s+s1}{w5}\PY{l+s+s1}{\PYZsq{}}\PY{p}{:}\PY{n}{tf}\PY{o}{.}\PY{n}{Variable}\PY{p}{(}\PY{n}{tf}\PY{o}{.}\PY{n}{random}\PY{o}{.}\PY{n}{truncated\PYZus{}normal}\PY{p}{(}\PY{p}{[}\PY{l+m+mi}{64}\PY{p}{,}\PY{l+m+mi}{32}\PY{p}{]}\PY{p}{,} \PY{n}{stddev}\PY{o}{=}\PY{l+m+mf}{0.1}\PY{p}{)}\PY{p}{)}\PY{p}{,}
    \PY{l+s+s1}{\PYZsq{}}\PY{l+s+s1}{w6}\PY{l+s+s1}{\PYZsq{}}\PY{p}{:}\PY{n}{tf}\PY{o}{.}\PY{n}{Variable}\PY{p}{(}\PY{n}{tf}\PY{o}{.}\PY{n}{random}\PY{o}{.}\PY{n}{truncated\PYZus{}normal}\PY{p}{(}\PY{p}{[}\PY{l+m+mi}{32}\PY{p}{,}\PY{l+m+mi}{16}\PY{p}{]}\PY{p}{,} \PY{n}{stddev}\PY{o}{=}\PY{l+m+mf}{0.1}\PY{p}{)}\PY{p}{)}\PY{p}{,}
    \PY{l+s+s1}{\PYZsq{}}\PY{l+s+s1}{out}\PY{l+s+s1}{\PYZsq{}}\PY{p}{:}\PY{n}{tf}\PY{o}{.}\PY{n}{Variable}\PY{p}{(}\PY{n}{tf}\PY{o}{.}\PY{n}{random}\PY{o}{.}\PY{n}{truncated\PYZus{}normal}\PY{p}{(}\PY{p}{[}\PY{l+m+mi}{16}\PY{p}{,}\PY{l+m+mi}{10}\PY{p}{]}\PY{p}{,} \PY{n}{stddev}\PY{o}{=}\PY{l+m+mf}{0.1}\PY{p}{)}\PY{p}{)}\PY{p}{,}
\PY{p}{\PYZcb{}}
\end{Verbatim}
\end{tcolorbox}

    \begin{tcolorbox}[breakable, size=fbox, boxrule=1pt, pad at break*=1mm,colback=cellbackground, colframe=cellborder]
\prompt{In}{incolor}{11}{\boxspacing}
\begin{Verbatim}[commandchars=\\\{\}]
\PY{n}{bias} \PY{o}{=} \PY{p}{\PYZob{}}
    \PY{l+s+s1}{\PYZsq{}}\PY{l+s+s1}{b1}\PY{l+s+s1}{\PYZsq{}}\PY{p}{:} \PY{n}{tf}\PY{o}{.}\PY{n}{Variable}\PY{p}{(}\PY{n}{tf}\PY{o}{.}\PY{n}{constant}\PY{p}{(}\PY{l+m+mf}{0.1}\PY{p}{,} \PY{n}{shape}\PY{o}{=}\PY{p}{[}\PY{l+m+mi}{256}\PY{p}{]}\PY{p}{)}\PY{p}{)}\PY{p}{,}
    \PY{l+s+s1}{\PYZsq{}}\PY{l+s+s1}{b2}\PY{l+s+s1}{\PYZsq{}}\PY{p}{:} \PY{n}{tf}\PY{o}{.}\PY{n}{Variable}\PY{p}{(}\PY{n}{tf}\PY{o}{.}\PY{n}{constant}\PY{p}{(}\PY{l+m+mf}{0.1}\PY{p}{,} \PY{n}{shape}\PY{o}{=}\PY{p}{[}\PY{l+m+mi}{256}\PY{p}{]}\PY{p}{)}\PY{p}{)}\PY{p}{,}
    \PY{l+s+s1}{\PYZsq{}}\PY{l+s+s1}{b3}\PY{l+s+s1}{\PYZsq{}}\PY{p}{:} \PY{n}{tf}\PY{o}{.}\PY{n}{Variable}\PY{p}{(}\PY{n}{tf}\PY{o}{.}\PY{n}{constant}\PY{p}{(}\PY{l+m+mf}{0.1}\PY{p}{,} \PY{n}{shape}\PY{o}{=}\PY{p}{[}\PY{l+m+mi}{128}\PY{p}{]}\PY{p}{)}\PY{p}{)}\PY{p}{,}
    \PY{l+s+s1}{\PYZsq{}}\PY{l+s+s1}{b4}\PY{l+s+s1}{\PYZsq{}}\PY{p}{:} \PY{n}{tf}\PY{o}{.}\PY{n}{Variable}\PY{p}{(}\PY{n}{tf}\PY{o}{.}\PY{n}{constant}\PY{p}{(}\PY{l+m+mf}{0.1}\PY{p}{,} \PY{n}{shape}\PY{o}{=}\PY{p}{[}\PY{l+m+mi}{64}\PY{p}{]}\PY{p}{)}\PY{p}{)}\PY{p}{,}
    \PY{l+s+s1}{\PYZsq{}}\PY{l+s+s1}{b5}\PY{l+s+s1}{\PYZsq{}}\PY{p}{:} \PY{n}{tf}\PY{o}{.}\PY{n}{Variable}\PY{p}{(}\PY{n}{tf}\PY{o}{.}\PY{n}{constant}\PY{p}{(}\PY{l+m+mf}{0.1}\PY{p}{,} \PY{n}{shape}\PY{o}{=}\PY{p}{[}\PY{l+m+mi}{32}\PY{p}{]}\PY{p}{)}\PY{p}{)}\PY{p}{,}
    \PY{l+s+s1}{\PYZsq{}}\PY{l+s+s1}{b6}\PY{l+s+s1}{\PYZsq{}}\PY{p}{:} \PY{n}{tf}\PY{o}{.}\PY{n}{Variable}\PY{p}{(}\PY{n}{tf}\PY{o}{.}\PY{n}{constant}\PY{p}{(}\PY{l+m+mf}{0.1}\PY{p}{,} \PY{n}{shape}\PY{o}{=}\PY{p}{[}\PY{l+m+mi}{16}\PY{p}{]}\PY{p}{)}\PY{p}{)}\PY{p}{,}
    \PY{l+s+s1}{\PYZsq{}}\PY{l+s+s1}{out}\PY{l+s+s1}{\PYZsq{}}\PY{p}{:} \PY{n}{tf}\PY{o}{.}\PY{n}{Variable}\PY{p}{(}\PY{n}{tf}\PY{o}{.}\PY{n}{constant}\PY{p}{(}\PY{l+m+mf}{0.1}\PY{p}{,} \PY{n}{shape}\PY{o}{=}\PY{p}{[}\PY{l+m+mi}{10}\PY{p}{]}\PY{p}{)}\PY{p}{)}
\PY{p}{\PYZcb{}}
\end{Verbatim}
\end{tcolorbox}

    \begin{tcolorbox}[breakable, size=fbox, boxrule=1pt, pad at break*=1mm,colback=cellbackground, colframe=cellborder]
\prompt{In}{incolor}{12}{\boxspacing}
\begin{Verbatim}[commandchars=\\\{\}]
\PY{c+c1}{\PYZsh{}\PYZsh{}\PYZsh{}capas}

\PY{n}{capa\PYZus{}1} \PY{o}{=} \PY{n}{tf}\PY{o}{.}\PY{n}{add}\PY{p}{(}\PY{n}{tf}\PY{o}{.}\PY{n}{matmul}\PY{p}{(}\PY{n}{X}\PY{p}{,} \PY{n}{w}\PY{p}{[}\PY{l+s+s1}{\PYZsq{}}\PY{l+s+s1}{w1}\PY{l+s+s1}{\PYZsq{}}\PY{p}{]}\PY{p}{)}\PY{p}{,} \PY{n}{bias}\PY{p}{[}\PY{l+s+s1}{\PYZsq{}}\PY{l+s+s1}{b1}\PY{l+s+s1}{\PYZsq{}}\PY{p}{]}\PY{p}{)}
\PY{n}{capa\PYZus{}2} \PY{o}{=} \PY{n}{tf}\PY{o}{.}\PY{n}{add}\PY{p}{(}\PY{n}{tf}\PY{o}{.}\PY{n}{matmul}\PY{p}{(}\PY{n}{capa\PYZus{}1}\PY{p}{,} \PY{n}{w}\PY{p}{[}\PY{l+s+s1}{\PYZsq{}}\PY{l+s+s1}{w2}\PY{l+s+s1}{\PYZsq{}}\PY{p}{]}\PY{p}{)}\PY{p}{,} \PY{n}{bias}\PY{p}{[}\PY{l+s+s1}{\PYZsq{}}\PY{l+s+s1}{b2}\PY{l+s+s1}{\PYZsq{}}\PY{p}{]}\PY{p}{)}
\PY{n}{capa\PYZus{}3} \PY{o}{=} \PY{n}{tf}\PY{o}{.}\PY{n}{add}\PY{p}{(}\PY{n}{tf}\PY{o}{.}\PY{n}{matmul}\PY{p}{(}\PY{n}{capa\PYZus{}2}\PY{p}{,} \PY{n}{w}\PY{p}{[}\PY{l+s+s1}{\PYZsq{}}\PY{l+s+s1}{w3}\PY{l+s+s1}{\PYZsq{}}\PY{p}{]}\PY{p}{)}\PY{p}{,} \PY{n}{bias}\PY{p}{[}\PY{l+s+s1}{\PYZsq{}}\PY{l+s+s1}{b3}\PY{l+s+s1}{\PYZsq{}}\PY{p}{]}\PY{p}{)}
\PY{n}{capa\PYZus{}4} \PY{o}{=} \PY{n}{tf}\PY{o}{.}\PY{n}{add}\PY{p}{(}\PY{n}{tf}\PY{o}{.}\PY{n}{matmul}\PY{p}{(}\PY{n}{capa\PYZus{}3}\PY{p}{,} \PY{n}{w}\PY{p}{[}\PY{l+s+s1}{\PYZsq{}}\PY{l+s+s1}{w4}\PY{l+s+s1}{\PYZsq{}}\PY{p}{]}\PY{p}{)}\PY{p}{,} \PY{n}{bias}\PY{p}{[}\PY{l+s+s1}{\PYZsq{}}\PY{l+s+s1}{b4}\PY{l+s+s1}{\PYZsq{}}\PY{p}{]}\PY{p}{)}
\PY{n}{capa\PYZus{}5} \PY{o}{=} \PY{n}{tf}\PY{o}{.}\PY{n}{add}\PY{p}{(}\PY{n}{tf}\PY{o}{.}\PY{n}{matmul}\PY{p}{(}\PY{n}{capa\PYZus{}4}\PY{p}{,} \PY{n}{w}\PY{p}{[}\PY{l+s+s1}{\PYZsq{}}\PY{l+s+s1}{w5}\PY{l+s+s1}{\PYZsq{}}\PY{p}{]}\PY{p}{)}\PY{p}{,} \PY{n}{bias}\PY{p}{[}\PY{l+s+s1}{\PYZsq{}}\PY{l+s+s1}{b5}\PY{l+s+s1}{\PYZsq{}}\PY{p}{]}\PY{p}{)}
\PY{n}{capa\PYZus{}6} \PY{o}{=} \PY{n}{tf}\PY{o}{.}\PY{n}{add}\PY{p}{(}\PY{n}{tf}\PY{o}{.}\PY{n}{matmul}\PY{p}{(}\PY{n}{capa\PYZus{}5}\PY{p}{,} \PY{n}{w}\PY{p}{[}\PY{l+s+s1}{\PYZsq{}}\PY{l+s+s1}{w6}\PY{l+s+s1}{\PYZsq{}}\PY{p}{]}\PY{p}{)}\PY{p}{,} \PY{n}{bias}\PY{p}{[}\PY{l+s+s1}{\PYZsq{}}\PY{l+s+s1}{b6}\PY{l+s+s1}{\PYZsq{}}\PY{p}{]}\PY{p}{)}
\PY{n}{capa\PYZus{}drop} \PY{o}{=} \PY{n}{tf}\PY{o}{.}\PY{n}{nn}\PY{o}{.}\PY{n}{dropout}\PY{p}{(}\PY{n}{capa\PYZus{}6}\PY{p}{,} \PY{n}{keep\PYZus{}prob}\PY{p}{)}
\PY{n}{capa\PYZus{}output} \PY{o}{=} \PY{n}{tf}\PY{o}{.}\PY{n}{matmul}\PY{p}{(}\PY{n}{capa\PYZus{}6}\PY{p}{,} \PY{n}{w}\PY{p}{[}\PY{l+s+s1}{\PYZsq{}}\PY{l+s+s1}{out}\PY{l+s+s1}{\PYZsq{}}\PY{p}{]}\PY{p}{)} \PY{o}{+} \PY{n}{bias}\PY{p}{[}\PY{l+s+s1}{\PYZsq{}}\PY{l+s+s1}{out}\PY{l+s+s1}{\PYZsq{}}\PY{p}{]}
\end{Verbatim}
\end{tcolorbox}

    \paragraph{Es necesario definir la funcion de perdida con el fin de
evaluar de forma precisa el comportamiento de la red al momento de
entrenar, para este caso se presenta la funcion cross\_entropy no
obstante en primeras experimentaciones se identifico que el modelo
quizas no era tan preciso como se pensaba por lo que se optó por
utilizar otra funcion de perdida **
insertar**}\label{es-necesario-definir-la-funcion-de-perdida-con-el-fin-de-evaluar-de-forma-precisa-el-comportamiento-de-la-red-al-momento-de-entrenar-para-este-caso-se-presenta-la-funcion-cross_entropy-no-obstante-en-primeras-experimentaciones-se-identifico-que-el-modelo-quizas-no-era-tan-preciso-como-se-pensaba-por-lo-que-se-optuxf3-por-utilizar-otra-funcion-de-perdida-insertar}

    \begin{tcolorbox}[breakable, size=fbox, boxrule=1pt, pad at break*=1mm,colback=cellbackground, colframe=cellborder]
\prompt{In}{incolor}{13}{\boxspacing}
\begin{Verbatim}[commandchars=\\\{\}]
\PY{n}{cross\PYZus{}entropy} \PY{o}{=} \PY{n}{tf}\PY{o}{.}\PY{n}{reduce\PYZus{}mean}\PY{p}{(}
                    \PY{n}{tf}\PY{o}{.}\PY{n}{nn}\PY{o}{.}\PY{n}{softmax\PYZus{}cross\PYZus{}entropy\PYZus{}with\PYZus{}logits}\PY{p}{(}
                        \PY{n}{labels}\PY{o}{=}\PY{n}{Y}\PY{p}{,} \PY{n}{logits} \PY{o}{=} \PY{n}{capa\PYZus{}output}
                    \PY{p}{)}
                \PY{p}{)}
\PY{n}{train\PYZus{}step} \PY{o}{=} \PY{n}{tf}\PY{o}{.}\PY{n}{compat}\PY{o}{.}\PY{n}{v1}\PY{o}{.}\PY{n}{train}\PY{o}{.}\PY{n}{AdamOptimizer}\PY{p}{(}\PY{l+m+mf}{0.00001}\PY{p}{)}\PY{o}{.}\PY{n}{minimize}\PY{p}{(}\PY{n}{cross\PYZus{}entropy}\PY{p}{)}
\end{Verbatim}
\end{tcolorbox}

    \section{Entrenamiento del modelo}\label{entrenamiento-del-modelo}

    \begin{tcolorbox}[breakable, size=fbox, boxrule=1pt, pad at break*=1mm,colback=cellbackground, colframe=cellborder]
\prompt{In}{incolor}{14}{\boxspacing}
\begin{Verbatim}[commandchars=\\\{\}]
\PY{n}{pred\PYZus{}correctas} \PY{o}{=} \PY{n}{tf}\PY{o}{.}\PY{n}{equal}\PY{p}{(}\PY{n}{tf}\PY{o}{.}\PY{n}{argmax}\PY{p}{(}\PY{n}{capa\PYZus{}output}\PY{p}{,} \PY{l+m+mi}{1}\PY{p}{)}\PY{p}{,} \PY{n}{tf}\PY{o}{.}\PY{n}{argmax}\PY{p}{(}\PY{n}{Y}\PY{p}{,}\PY{l+m+mi}{1}\PY{p}{)}\PY{p}{)}
\PY{n}{accuracy} \PY{o}{=} \PY{n}{tf}\PY{o}{.}\PY{n}{reduce\PYZus{}mean}\PY{p}{(}\PY{n}{tf}\PY{o}{.}\PY{n}{cast}\PY{p}{(}\PY{n}{pred\PYZus{}correctas}\PY{p}{,} \PY{n}{tf}\PY{o}{.}\PY{n}{float32}\PY{p}{)}\PY{p}{)}
\end{Verbatim}
\end{tcolorbox}

    \begin{tcolorbox}[breakable, size=fbox, boxrule=1pt, pad at break*=1mm,colback=cellbackground, colframe=cellborder]
\prompt{In}{incolor}{15}{\boxspacing}
\begin{Verbatim}[commandchars=\\\{\}]
\PY{n}{init} \PY{o}{=} \PY{n}{tf}\PY{o}{.}\PY{n}{compat}\PY{o}{.}\PY{n}{v1}\PY{o}{.}\PY{n}{global\PYZus{}variables\PYZus{}initializer}\PY{p}{(}\PY{p}{)}
\PY{n}{sess} \PY{o}{=} \PY{n}{tf}\PY{o}{.}\PY{n}{compat}\PY{o}{.}\PY{n}{v1}\PY{o}{.}\PY{n}{Session}\PY{p}{(}\PY{p}{)}
\PY{n}{sess}\PY{o}{.}\PY{n}{run}\PY{p}{(}\PY{n}{init}\PY{p}{)}
\end{Verbatim}
\end{tcolorbox}

    \paragraph{El proceso de entrenamiento en batches se basa en seccionar
el conjunto total de datos de entrenamiento en un numero arbitratrio de
elementos, acto seguido y de forma aleatoria se selecciona uno de los
subconjuntos y se realiza el entrenamiento del modelo, todo esto para
cada iteracion. Cabe resaltar que la cantidad de iteraciones tambien
puede contribuir a la perdida de precision del modelo. Asi mismo se
intenta el ejercicio de entrenar el modelo sin particionar el dataset y
disminuyendo la cantidad de iteraciones en busqueda de mejores
resultados en la
evaluación}\label{el-proceso-de-entrenamiento-en-batches-se-basa-en-seccionar-el-conjunto-total-de-datos-de-entrenamiento-en-un-numero-arbitratrio-de-elementos-acto-seguido-y-de-forma-aleatoria-se-selecciona-uno-de-los-subconjuntos-y-se-realiza-el-entrenamiento-del-modelo-todo-esto-para-cada-iteracion.-cabe-resaltar-que-la-cantidad-de-iteraciones-tambien-puede-contribuir-a-la-perdida-de-precision-del-modelo.-asi-mismo-se-intenta-el-ejercicio-de-entrenar-el-modelo-sin-particionar-el-dataset-y-disminuyendo-la-cantidad-de-iteraciones-en-busqueda-de-mejores-resultados-en-la-evaluaciuxf3n}

    \begin{tcolorbox}[breakable, size=fbox, boxrule=1pt, pad at break*=1mm,colback=cellbackground, colframe=cellborder]
\prompt{In}{incolor}{16}{\boxspacing}
\begin{Verbatim}[commandchars=\\\{\}]
\PY{c+c1}{\PYZsh{}\PYZsh{}\PYZsh{} entrenamiento en batches (bloques)}
\PY{n}{learning\PYZus{}rate} \PY{o}{=} \PY{l+m+mf}{1e\PYZhy{}4}
\PY{n}{n\PYZus{}iter} \PY{o}{=} \PY{l+m+mi}{10000}
\PY{n}{batch\PYZus{}size} \PY{o}{=} \PY{l+m+mi}{256}
\PY{n}{dropout} \PY{o}{=} \PY{l+m+mf}{0.3}

\PY{n}{cantidad\PYZus{}batches} \PY{o}{=} \PY{n}{math}\PY{o}{.}\PY{n}{ceil}\PY{p}{(}\PY{n+nb}{len}\PY{p}{(}\PY{n}{train\PYZus{}images}\PY{p}{)}\PY{o}{/}\PY{n}{batch\PYZus{}size}\PY{p}{)}
\PY{n}{train\PYZus{}images\PYZus{}plain} \PY{o}{=} \PY{p}{[}\PY{p}{]}

\PY{k}{for} \PY{n}{i} \PY{o+ow}{in} \PY{n+nb}{range}\PY{p}{(}\PY{n+nb}{len}\PY{p}{(}\PY{n}{train\PYZus{}images}\PY{p}{)}\PY{p}{)}\PY{p}{:}
    \PY{n}{train\PYZus{}images\PYZus{}plain}\PY{o}{.}\PY{n}{append}\PY{p}{(}\PY{n}{train\PYZus{}images}\PY{p}{[}\PY{n}{i}\PY{p}{]}\PY{o}{.}\PY{n}{flatten}\PY{p}{(}\PY{p}{)}\PY{p}{)}


\PY{n}{train\PYZus{}images\PYZus{}batch} \PY{o}{=} \PY{n}{np}\PY{o}{.}\PY{n}{array\PYZus{}split}\PY{p}{(}\PY{n}{train\PYZus{}images\PYZus{}plain}\PY{p}{,} \PY{n}{cantidad\PYZus{}batches}\PY{p}{)}
\PY{n}{train\PYZus{}labels\PYZus{}batch} \PY{o}{=} \PY{n}{np}\PY{o}{.}\PY{n}{array\PYZus{}split}\PY{p}{(}\PY{n}{train\PYZus{}labels\PYZus{}OH}\PY{p}{,} \PY{n}{cantidad\PYZus{}batches}\PY{p}{)}


\PY{n}{idx} \PY{o}{=} \PY{n}{np}\PY{o}{.}\PY{n}{arange}\PY{p}{(}\PY{l+m+mi}{0} \PY{p}{,} \PY{n+nb}{len}\PY{p}{(}\PY{n}{train\PYZus{}images\PYZus{}batch}\PY{p}{)}\PY{p}{)}
\PY{n}{outputs\PYZus{}loss}\PY{o}{=}\PY{p}{[}\PY{p}{]}
\PY{n}{outputs\PYZus{}acc}\PY{o}{=}\PY{p}{[}\PY{p}{]}
\PY{n}{itera}\PY{o}{=}\PY{p}{[}\PY{p}{]}

\PY{k}{for} \PY{n}{i} \PY{o+ow}{in} \PY{n+nb}{range}\PY{p}{(}\PY{n}{n\PYZus{}iter}\PY{p}{)}\PY{p}{:}
    \PY{n}{np}\PY{o}{.}\PY{n}{random}\PY{o}{.}\PY{n}{shuffle}\PY{p}{(}\PY{n}{idx}\PY{p}{)}
    \PY{n}{k}\PY{o}{=}\PY{n}{idx}\PY{p}{[}\PY{l+m+mi}{0}\PY{p}{]}
    \PY{n}{batch\PYZus{}x} \PY{o}{=} \PY{n}{train\PYZus{}images\PYZus{}batch}\PY{p}{[}\PY{n}{k}\PY{p}{]}
    \PY{n}{batch\PYZus{}y} \PY{o}{=} \PY{n}{train\PYZus{}labels\PYZus{}batch}\PY{p}{[}\PY{n}{k}\PY{p}{]}
    \PY{n}{sess}\PY{o}{.}\PY{n}{run}\PY{p}{(}\PY{n}{train\PYZus{}step}\PY{p}{,} \PY{n}{feed\PYZus{}dict}\PY{o}{=}\PY{p}{\PYZob{}}
        \PY{n}{X}\PY{p}{:}\PY{n}{batch\PYZus{}x}\PY{p}{,} \PY{n}{Y}\PY{p}{:}\PY{n}{batch\PYZus{}y}\PY{p}{,} \PY{n}{keep\PYZus{}prob}\PY{p}{:}\PY{n}{dropout}
    \PY{p}{\PYZcb{}}\PY{p}{)}
    
    \PY{k}{if} \PY{n}{i}\PY{o}{\PYZpc{}}\PY{k}{500} == 0:
        \PY{n}{minibatch\PYZus{}loss}\PY{p}{,} \PY{n}{minibatch\PYZus{}accuracy} \PY{o}{=} \PY{n}{sess}\PY{o}{.}\PY{n}{run}\PY{p}{(}
            \PY{p}{[}\PY{n}{cross\PYZus{}entropy}\PY{p}{,} \PY{n}{accuracy}\PY{p}{]}\PY{p}{,}
            \PY{n}{feed\PYZus{}dict}\PY{o}{=}\PY{p}{\PYZob{}}\PY{n}{X}\PY{p}{:} \PY{n}{batch\PYZus{}x}\PY{p}{,} \PY{n}{Y}\PY{p}{:} \PY{n}{batch\PYZus{}y}\PY{p}{,} \PY{n}{keep\PYZus{}prob}\PY{p}{:} \PY{l+m+mf}{1.0}\PY{p}{\PYZcb{}}
            \PY{p}{)}
        \PY{n}{val} \PY{o}{=} \PY{p}{[}\PY{n}{minibatch\PYZus{}loss}\PY{p}{,}\PY{n}{minibatch\PYZus{}accuracy}\PY{p}{]}
        \PY{n}{outputs\PYZus{}loss}\PY{o}{.}\PY{n}{append}\PY{p}{(}\PY{n}{val}\PY{p}{[}\PY{l+m+mi}{0}\PY{p}{]}\PY{p}{)}
        \PY{n}{outputs\PYZus{}acc}\PY{o}{.}\PY{n}{append}\PY{p}{(}\PY{n}{val}\PY{p}{[}\PY{l+m+mi}{1}\PY{p}{]}\PY{p}{)}
        \PY{n}{itera}\PY{o}{.}\PY{n}{append}\PY{p}{(}\PY{n}{i}\PY{p}{)}
        
        \PY{n+nb}{print}\PY{p}{(}\PY{l+s+s2}{\PYZdq{}}\PY{l+s+s2}{Iter: }\PY{l+s+s2}{\PYZdq{}}\PY{p}{,}\PY{n+nb}{str}\PY{p}{(}\PY{n}{i}\PY{p}{)}\PY{p}{,}\PY{l+s+s2}{\PYZdq{}}\PY{l+s+se}{\PYZbs{}t}\PY{l+s+s2}{| Loss =}\PY{l+s+s2}{\PYZdq{}}\PY{p}{,}\PY{n+nb}{str}\PY{p}{(}\PY{n}{minibatch\PYZus{}loss}\PY{p}{)}\PY{p}{,}\PY{l+s+s2}{\PYZdq{}}\PY{l+s+se}{\PYZbs{}t}\PY{l+s+s2}{| Accuracy =}\PY{l+s+s2}{\PYZdq{}}\PY{p}{,}\PY{n+nb}{str}\PY{p}{(}\PY{n}{minibatch\PYZus{}accuracy}\PY{p}{)}\PY{p}{)}
\end{Verbatim}
\end{tcolorbox}

    \begin{Verbatim}[commandchars=\\\{\}]
Iter:  0        | Loss = 2.3313487      | Accuracy = 0.08235294
Iter:  500      | Loss = 1.5391095      | Accuracy = 0.58203125
Iter:  1000     | Loss = 0.9563483      | Accuracy = 0.76953125
Iter:  1500     | Loss = 0.71808845     | Accuracy = 0.8039216
Iter:  2000     | Loss = 0.5843232      | Accuracy = 0.83984375
Iter:  2500     | Loss = 0.54150295     | Accuracy = 0.85490197
Iter:  3000     | Loss = 0.3821045      | Accuracy = 0.8862745
Iter:  3500     | Loss = 0.4262028      | Accuracy = 0.88671875
Iter:  4000     | Loss = 0.63882715     | Accuracy = 0.84375
Iter:  4500     | Loss = 0.2137761      | Accuracy = 0.9411765
Iter:  5000     | Loss = 0.3392076      | Accuracy = 0.8828125
Iter:  5500     | Loss = 0.26856056     | Accuracy = 0.90588236
Iter:  6000     | Loss = 0.46691394     | Accuracy = 0.8509804
Iter:  6500     | Loss = 0.36758706     | Accuracy = 0.8901961
Iter:  7000     | Loss = 0.30143458     | Accuracy = 0.9019608
Iter:  7500     | Loss = 0.5752704      | Accuracy = 0.8515625
Iter:  8000     | Loss = 0.309514       | Accuracy = 0.92156863
Iter:  8500     | Loss = 0.25214913     | Accuracy = 0.93333334
Iter:  9000     | Loss = 0.2253856      | Accuracy = 0.9375
Iter:  9500     | Loss = 0.3537338      | Accuracy = 0.9137255
    \end{Verbatim}

    \paragraph{Observando la tabla anterior vemos como los presentan
crecimiento hasta un punto determinado en el cual empiezan a fluctuar,
esta fluctuacion es producto de la ultilizacion del entrenamiento en
batch}\label{observando-la-tabla-anterior-vemos-como-los-presentan-crecimiento-hasta-un-punto-determinado-en-el-cual-empiezan-a-fluctuar-esta-fluctuacion-es-producto-de-la-ultilizacion-del-entrenamiento-en-batch}

    \begin{tcolorbox}[breakable, size=fbox, boxrule=1pt, pad at break*=1mm,colback=cellbackground, colframe=cellborder]
\prompt{In}{incolor}{17}{\boxspacing}
\begin{Verbatim}[commandchars=\\\{\}]
\PY{n}{fig}\PY{p}{,} \PY{n}{ax} \PY{o}{=} \PY{n}{plt}\PY{o}{.}\PY{n}{subplots}\PY{p}{(}\PY{p}{)}

\PY{n}{ax}\PY{o}{.}\PY{n}{plot}\PY{p}{(}\PY{n}{itera}\PY{p}{,}\PY{n}{outputs\PYZus{}acc}\PY{p}{,}\PY{n}{color} \PY{o}{=} \PY{l+s+s1}{\PYZsq{}}\PY{l+s+s1}{green}\PY{l+s+s1}{\PYZsq{}}\PY{p}{,} \PY{n}{label} \PY{o}{=} \PY{l+s+s1}{\PYZsq{}}\PY{l+s+s1}{Accuracy}\PY{l+s+s1}{\PYZsq{}}\PY{p}{)}
\PY{n}{ax}\PY{o}{.}\PY{n}{plot}\PY{p}{(}\PY{n}{itera}\PY{p}{,}\PY{n}{outputs\PYZus{}loss}\PY{p}{,} \PY{n}{color} \PY{o}{=} \PY{l+s+s1}{\PYZsq{}}\PY{l+s+s1}{red}\PY{l+s+s1}{\PYZsq{}}\PY{p}{,} \PY{n}{label} \PY{o}{=} \PY{l+s+s1}{\PYZsq{}}\PY{l+s+s1}{loss}\PY{l+s+s1}{\PYZsq{}}\PY{p}{)}
\PY{n}{ax}\PY{o}{.}\PY{n}{legend}\PY{p}{(}\PY{n}{loc} \PY{o}{=} \PY{l+s+s1}{\PYZsq{}}\PY{l+s+s1}{upper left}\PY{l+s+s1}{\PYZsq{}}\PY{p}{)}
\PY{n}{plt}\PY{o}{.}\PY{n}{show}\PY{p}{(}\PY{p}{)}
\end{Verbatim}
\end{tcolorbox}

    \begin{center}
    \adjustimage{max size={0.9\linewidth}{0.9\paperheight}}{output_29_0.png}
    \end{center}
    { \hspace*{\fill} \\}
    
    \begin{tcolorbox}[breakable, size=fbox, boxrule=1pt, pad at break*=1mm,colback=cellbackground, colframe=cellborder]
\prompt{In}{incolor}{18}{\boxspacing}
\begin{Verbatim}[commandchars=\\\{\}]
\PY{n}{test\PYZus{}images\PYZus{}plain} \PY{o}{=}\PY{p}{[}\PY{p}{]}
\PY{k}{for} \PY{n}{i} \PY{o+ow}{in} \PY{n+nb}{range}\PY{p}{(}\PY{n+nb}{len}\PY{p}{(}\PY{n}{test\PYZus{}images}\PY{p}{)}\PY{p}{)}\PY{p}{:}
    \PY{n}{test\PYZus{}images\PYZus{}plain}\PY{o}{.}\PY{n}{append}\PY{p}{(}\PY{n}{test\PYZus{}images}\PY{p}{[}\PY{n}{i}\PY{p}{]}\PY{o}{.}\PY{n}{flatten}\PY{p}{(}\PY{p}{)}\PY{p}{)}
    

\PY{n}{test\PYZus{}accuracy} \PY{o}{=} \PY{n}{sess}\PY{o}{.}\PY{n}{run}\PY{p}{(}\PY{n}{accuracy}\PY{p}{,} \PY{n}{feed\PYZus{}dict}\PY{o}{=}\PY{p}{\PYZob{}}\PY{n}{X}\PY{p}{:} \PY{n}{test\PYZus{}images\PYZus{}plain}\PY{p}{,} \PY{n}{Y}\PY{p}{:} \PY{n}{test\PYZus{}labels\PYZus{}OH}\PY{p}{,} \PY{n}{keep\PYZus{}prob}\PY{p}{:}\PY{l+m+mf}{1.0}\PY{p}{\PYZcb{}}\PY{p}{)}
\PY{n+nb}{print}\PY{p}{(}\PY{l+s+s2}{\PYZdq{}}\PY{l+s+se}{\PYZbs{}n}\PY{l+s+s2}{Accuracy on test set:}\PY{l+s+s2}{\PYZdq{}}\PY{p}{,} \PY{n}{test\PYZus{}accuracy}\PY{p}{)}
\end{Verbatim}
\end{tcolorbox}

    \begin{Verbatim}[commandchars=\\\{\}]

Accuracy on test set: 0.9151151
    \end{Verbatim}

    \section{Pruebas con datos reales}\label{pruebas-con-datos-reales}

    \begin{tcolorbox}[breakable, size=fbox, boxrule=1pt, pad at break*=1mm,colback=cellbackground, colframe=cellborder]
\prompt{In}{incolor}{19}{\boxspacing}
\begin{Verbatim}[commandchars=\\\{\}]
\PY{n}{plt}\PY{o}{.}\PY{n}{figure}\PY{p}{(}\PY{n}{figsize}\PY{o}{=}\PY{p}{(}\PY{l+m+mi}{8}\PY{p}{,}\PY{l+m+mi}{3}\PY{p}{)}\PY{p}{)}
\PY{k}{for} \PY{n}{i} \PY{o+ow}{in} \PY{n+nb}{range}\PY{p}{(}\PY{l+m+mi}{10}\PY{p}{)}\PY{p}{:}
    \PY{n}{plt}\PY{o}{.}\PY{n}{subplot}\PY{p}{(}\PY{l+m+mi}{2}\PY{p}{,}\PY{l+m+mi}{5}\PY{p}{,}\PY{n}{i}\PY{o}{+}\PY{l+m+mi}{1}\PY{p}{)}
    \PY{n}{plt}\PY{o}{.}\PY{n}{xticks}\PY{p}{(}\PY{p}{[}\PY{p}{]}\PY{p}{)}
    \PY{n}{plt}\PY{o}{.}\PY{n}{yticks}\PY{p}{(}\PY{p}{[}\PY{p}{]}\PY{p}{)}
    \PY{n}{plt}\PY{o}{.}\PY{n}{grid}\PY{p}{(}\PY{k+kc}{False}\PY{p}{)}
    \PY{n}{k} \PY{o}{=} \PY{n}{random}\PY{o}{.}\PY{n}{randint}\PY{p}{(}\PY{l+m+mi}{0}\PY{p}{,} \PY{n+nb}{len}\PY{p}{(}\PY{n}{train\PYZus{}images}\PY{p}{)}\PY{p}{)}
    \PY{n}{plt}\PY{o}{.}\PY{n}{imshow}\PY{p}{(}\PY{n}{val\PYZus{}images}\PY{p}{[}\PY{n}{i}\PY{p}{]}\PY{p}{,} \PY{n}{cmap}\PY{o}{=}\PY{n}{plt}\PY{o}{.}\PY{n}{cm}\PY{o}{.}\PY{n}{gray}\PY{p}{)}
    \PY{n}{plt}\PY{o}{.}\PY{n}{xlabel}\PY{p}{(}\PY{n}{val\PYZus{}labels}\PY{p}{[}\PY{n}{i}\PY{p}{]}\PY{p}{)}
\PY{n}{plt}\PY{o}{.}\PY{n}{show}\PY{p}{(}\PY{p}{)}
\end{Verbatim}
\end{tcolorbox}

    \begin{center}
    \adjustimage{max size={0.9\linewidth}{0.9\paperheight}}{output_32_0.png}
    \end{center}
    { \hspace*{\fill} \\}
    
    \begin{tcolorbox}[breakable, size=fbox, boxrule=1pt, pad at break*=1mm,colback=cellbackground, colframe=cellborder]
\prompt{In}{incolor}{20}{\boxspacing}
\begin{Verbatim}[commandchars=\\\{\}]
\PY{k}{for} \PY{n}{i} \PY{o+ow}{in} \PY{n+nb}{range}\PY{p}{(}\PY{n+nb}{len}\PY{p}{(}\PY{n}{val\PYZus{}images}\PY{p}{)}\PY{p}{)}\PY{p}{:}
    \PY{n}{prediction} \PY{o}{=} \PY{n}{sess}\PY{o}{.}\PY{n}{run}\PY{p}{(}\PY{n}{tf}\PY{o}{.}\PY{n}{argmax}\PY{p}{(}\PY{n}{capa\PYZus{}output}\PY{p}{,} \PY{l+m+mi}{1}\PY{p}{)}\PY{p}{,} \PY{n}{feed\PYZus{}dict}\PY{o}{=}\PY{p}{\PYZob{}}\PY{n}{X}\PY{p}{:} \PY{p}{[}\PY{n}{val\PYZus{}images}\PY{p}{[}\PY{n}{i}\PY{p}{]}\PY{o}{.}\PY{n}{flatten}\PY{p}{(}\PY{p}{)}\PY{p}{]}\PY{p}{\PYZcb{}}\PY{p}{)}
    \PY{n+nb}{print} \PY{p}{(}\PY{l+s+s2}{\PYZdq{}}\PY{l+s+s2}{Prediccion de la imagen de prueba}\PY{l+s+s2}{\PYZdq{}}\PY{p}{,} \PY{n}{np}\PY{o}{.}\PY{n}{squeeze}\PY{p}{(}\PY{n}{prediction}\PY{p}{)}\PY{p}{)}
    \PY{n+nb}{print} \PY{p}{(}\PY{l+s+s2}{\PYZdq{}}\PY{l+s+s2}{Valor real tomada de las etiquetas: }\PY{l+s+s2}{\PYZdq{}}\PY{p}{,}\PY{n}{val\PYZus{}labels}\PY{p}{[}\PY{n}{i}\PY{p}{]}\PY{p}{)}
\end{Verbatim}
\end{tcolorbox}

    \begin{Verbatim}[commandchars=\\\{\}]
Prediccion de la imagen de prueba 7
Valor real tomada de las etiquetas:  7
Prediccion de la imagen de prueba 8
Valor real tomada de las etiquetas:  8
Prediccion de la imagen de prueba 9
Valor real tomada de las etiquetas:  9
Prediccion de la imagen de prueba 0
Valor real tomada de las etiquetas:  0
Prediccion de la imagen de prueba 1
Valor real tomada de las etiquetas:  1
Prediccion de la imagen de prueba 2
Valor real tomada de las etiquetas:  2
Prediccion de la imagen de prueba 3
Valor real tomada de las etiquetas:  3
Prediccion de la imagen de prueba 4
Valor real tomada de las etiquetas:  4
Prediccion de la imagen de prueba 5
Valor real tomada de las etiquetas:  5
Prediccion de la imagen de prueba 6
Valor real tomada de las etiquetas:  6
    \end{Verbatim}

    \begin{tcolorbox}[breakable, size=fbox, boxrule=1pt, pad at break*=1mm,colback=cellbackground, colframe=cellborder]
\prompt{In}{incolor}{29}{\boxspacing}
\begin{Verbatim}[commandchars=\\\{\}]
\PY{n}{names} \PY{o}{=} \PY{p}{[}\PY{l+s+s1}{\PYZsq{}}\PY{l+s+s1}{1m}\PY{l+s+s1}{\PYZsq{}}\PY{p}{,}\PY{l+s+s1}{\PYZsq{}}\PY{l+s+s1}{1v2m}\PY{l+s+s1}{\PYZsq{}}\PY{p}{,}\PY{l+s+s1}{\PYZsq{}}\PY{l+s+s1}{3m}\PY{l+s+s1}{\PYZsq{}}\PY{p}{,}\PY{l+s+s1}{\PYZsq{}}\PY{l+s+s1}{4V1m}\PY{l+s+s1}{\PYZsq{}}\PY{p}{,}\PY{l+s+s1}{\PYZsq{}}\PY{l+s+s1}{4V2m}\PY{l+s+s1}{\PYZsq{}}\PY{p}{,}\PY{l+s+s1}{\PYZsq{}}\PY{l+s+s1}{5m}\PY{l+s+s1}{\PYZsq{}}\PY{p}{,}\PY{l+s+s1}{\PYZsq{}}\PY{l+s+s1}{8m}\PY{l+s+s1}{\PYZsq{}}\PY{p}{,}\PY{l+s+s1}{\PYZsq{}}\PY{l+s+s1}{9m}\PY{l+s+s1}{\PYZsq{}}\PY{p}{]}
\PY{n}{values} \PY{o}{=} \PY{p}{[}\PY{l+m+mi}{1}\PY{p}{,}\PY{l+m+mi}{1}\PY{p}{,}\PY{l+m+mi}{3}\PY{p}{,}\PY{l+m+mi}{4}\PY{p}{,}\PY{l+m+mi}{4}\PY{p}{,}\PY{l+m+mi}{5}\PY{p}{,}\PY{l+m+mi}{8}\PY{p}{,}\PY{l+m+mi}{9}\PY{p}{]}
\PY{n}{imgs} \PY{o}{=} \PY{p}{[}\PY{p}{]}
\PY{n}{inverts1} \PY{o}{=} \PY{p}{[}\PY{p}{]}
\PY{n}{inverts2} \PY{o}{=} \PY{p}{[}\PY{p}{]}
\PY{k}{for} \PY{n}{name} \PY{o+ow}{in} \PY{n}{names}\PY{p}{:}
    \PY{n}{imgs}\PY{o}{.}\PY{n}{append}\PY{p}{(}\PY{n}{Image}\PY{o}{.}\PY{n}{open}\PY{p}{(}\PY{n}{name}\PY{o}{+}\PY{l+s+s1}{\PYZsq{}}\PY{l+s+s1}{.jpg}\PY{l+s+s1}{\PYZsq{}}\PY{p}{)}\PY{p}{)}


\PY{k}{for} \PY{n}{img} \PY{o+ow}{in} \PY{n}{imgs}\PY{p}{:}
    \PY{n}{invertA} \PY{o}{=} \PY{n}{ImageOps}\PY{o}{.}\PY{n}{invert}\PY{p}{(}\PY{n}{img}\PY{o}{.}\PY{n}{convert}\PY{p}{(}\PY{l+s+s1}{\PYZsq{}}\PY{l+s+s1}{RGB}\PY{l+s+s1}{\PYZsq{}}\PY{p}{)}\PY{p}{)}
    \PY{n}{invertB} \PY{o}{=} \PY{n}{np}\PY{o}{.}\PY{n}{invert}\PY{p}{(}\PY{n}{img}\PY{o}{.}\PY{n}{convert}\PY{p}{(}\PY{l+s+s1}{\PYZsq{}}\PY{l+s+s1}{L}\PY{l+s+s1}{\PYZsq{}}\PY{p}{)}\PY{p}{)}
    \PY{n}{inverts1}\PY{o}{.}\PY{n}{append}\PY{p}{(}\PY{n}{invertA}\PY{p}{)}
    \PY{n}{inverts2}\PY{o}{.}\PY{n}{append}\PY{p}{(}\PY{n}{invertB}\PY{p}{)}
    

\PY{n}{imgs} \PY{o}{=} \PY{n}{inverts1}
    

\PY{n}{plt}\PY{o}{.}\PY{n}{figure}\PY{p}{(}\PY{n}{figsize}\PY{o}{=}\PY{p}{(}\PY{l+m+mi}{8}\PY{p}{,}\PY{l+m+mi}{3}\PY{p}{)}\PY{p}{)}
\PY{k}{for} \PY{n}{i} \PY{o+ow}{in} \PY{n+nb}{range}\PY{p}{(}\PY{l+m+mi}{8}\PY{p}{)}\PY{p}{:}
    \PY{n}{plt}\PY{o}{.}\PY{n}{subplot}\PY{p}{(}\PY{l+m+mi}{2}\PY{p}{,}\PY{l+m+mi}{4}\PY{p}{,}\PY{n}{i}\PY{o}{+}\PY{l+m+mi}{1}\PY{p}{)}
    \PY{n}{plt}\PY{o}{.}\PY{n}{xticks}\PY{p}{(}\PY{p}{[}\PY{p}{]}\PY{p}{)}
    \PY{n}{plt}\PY{o}{.}\PY{n}{yticks}\PY{p}{(}\PY{p}{[}\PY{p}{]}\PY{p}{)}
    \PY{n}{plt}\PY{o}{.}\PY{n}{grid}\PY{p}{(}\PY{k+kc}{False}\PY{p}{)}
    \PY{n}{k} \PY{o}{=} \PY{n}{random}\PY{o}{.}\PY{n}{randint}\PY{p}{(}\PY{l+m+mi}{0}\PY{p}{,} \PY{n+nb}{len}\PY{p}{(}\PY{n}{train\PYZus{}images}\PY{p}{)}\PY{p}{)}
    \PY{n}{plt}\PY{o}{.}\PY{n}{imshow}\PY{p}{(}\PY{n}{imgs}\PY{p}{[}\PY{n}{i}\PY{p}{]}\PY{p}{,} \PY{n}{cmap}\PY{o}{=}\PY{n}{plt}\PY{o}{.}\PY{n}{cm}\PY{o}{.}\PY{n}{gray}\PY{p}{)}
    \PY{n}{plt}\PY{o}{.}\PY{n}{xlabel}\PY{p}{(}\PY{n}{values}\PY{p}{[}\PY{n}{i}\PY{p}{]}\PY{p}{)}
\PY{n}{plt}\PY{o}{.}\PY{n}{show}\PY{p}{(}\PY{p}{)}
\end{Verbatim}
\end{tcolorbox}

    \begin{center}
    \adjustimage{max size={0.9\linewidth}{0.9\paperheight}}{output_34_0.png}
    \end{center}
    { \hspace*{\fill} \\}
    
    \begin{tcolorbox}[breakable, size=fbox, boxrule=1pt, pad at break*=1mm,colback=cellbackground, colframe=cellborder]
\prompt{In}{incolor}{32}{\boxspacing}
\begin{Verbatim}[commandchars=\\\{\}]
\PY{k}{for} \PY{n}{i} \PY{o+ow}{in} \PY{n+nb}{range}\PY{p}{(}\PY{n+nb}{len}\PY{p}{(}\PY{n}{imgs}\PY{p}{)}\PY{p}{)}\PY{p}{:}
    \PY{n}{prediction} \PY{o}{=} \PY{n}{sess}\PY{o}{.}\PY{n}{run}\PY{p}{(}\PY{n}{tf}\PY{o}{.}\PY{n}{argmax}\PY{p}{(}\PY{n}{capa\PYZus{}output}\PY{p}{,} \PY{l+m+mi}{1}\PY{p}{)}\PY{p}{,} \PY{n}{feed\PYZus{}dict}\PY{o}{=}\PY{p}{\PYZob{}}\PY{n}{X}\PY{p}{:} \PY{p}{[}\PY{n}{inverts2}\PY{p}{[}\PY{n}{i}\PY{p}{]}\PY{o}{.}\PY{n}{ravel}\PY{p}{(}\PY{p}{)}\PY{p}{]}\PY{p}{\PYZcb{}}\PY{p}{)}
    \PY{n}{valor}\PY{o}{=}\PY{l+s+s1}{\PYZsq{}}\PY{l+s+s1}{No acertado}\PY{l+s+s1}{\PYZsq{}}
    
    \PY{k}{if}\PY{p}{(}\PY{n}{np}\PY{o}{.}\PY{n}{squeeze}\PY{p}{(}\PY{n}{prediction}\PY{p}{)} \PY{o}{==} \PY{n}{values}\PY{p}{[}\PY{n}{i}\PY{p}{]}\PY{p}{)}\PY{p}{:}
        \PY{n}{valor} \PY{o}{=} \PY{l+s+s1}{\PYZsq{}}\PY{l+s+s1}{Acertado}\PY{l+s+s1}{\PYZsq{}}
    
    \PY{n+nb}{print} \PY{p}{(}\PY{l+s+s2}{\PYZdq{}}\PY{l+s+s2}{Prediccion de la imagen de prueba}\PY{l+s+s2}{\PYZdq{}}\PY{p}{,} \PY{n}{np}\PY{o}{.}\PY{n}{squeeze}\PY{p}{(}\PY{n}{prediction}\PY{p}{)}\PY{p}{,} \PY{n}{valor}\PY{p}{)}
    \PY{n+nb}{print} \PY{p}{(}\PY{l+s+s2}{\PYZdq{}}\PY{l+s+s2}{Valor real tomada de las etiquetas: }\PY{l+s+s2}{\PYZdq{}}\PY{p}{,}\PY{n}{values}\PY{p}{[}\PY{n}{i}\PY{p}{]}\PY{p}{)}
    \PY{n+nb}{print} \PY{p}{(}\PY{l+s+s2}{\PYZdq{}}\PY{l+s+s2}{\PYZhy{}\PYZhy{}\PYZhy{}\PYZhy{}\PYZhy{}\PYZhy{}\PYZhy{}\PYZhy{}\PYZhy{}\PYZhy{}\PYZhy{}\PYZhy{}\PYZhy{}\PYZhy{}\PYZhy{}\PYZhy{}\PYZhy{}\PYZhy{}\PYZhy{}\PYZhy{}\PYZhy{}\PYZhy{}\PYZhy{}\PYZhy{}\PYZhy{}\PYZhy{}\PYZhy{}\PYZhy{}\PYZhy{}\PYZhy{}\PYZhy{}\PYZhy{}\PYZhy{}\PYZhy{}\PYZhy{}\PYZhy{}\PYZhy{}\PYZhy{}\PYZhy{}\PYZhy{}\PYZhy{}\PYZhy{}\PYZhy{}\PYZhy{}\PYZhy{}\PYZhy{}\PYZhy{}\PYZhy{}\PYZhy{}\PYZhy{}\PYZhy{}\PYZhy{}\PYZhy{}\PYZhy{}\PYZhy{}\PYZhy{}\PYZhy{}\PYZhy{}\PYZhy{}\PYZhy{}\PYZhy{}\PYZhy{}\PYZhy{}\PYZhy{}\PYZhy{}\PYZhy{}\PYZhy{}\PYZhy{}\PYZhy{}\PYZhy{}\PYZhy{}\PYZhy{}\PYZhy{}}\PY{l+s+s2}{\PYZdq{}}\PY{p}{)}
\end{Verbatim}
\end{tcolorbox}

    \begin{Verbatim}[commandchars=\\\{\}]
Prediccion de la imagen de prueba 2 No acertado
Valor real tomada de las etiquetas:  1
-------------------------------------------------------------------------
Prediccion de la imagen de prueba 1 Acertado
Valor real tomada de las etiquetas:  1
-------------------------------------------------------------------------
Prediccion de la imagen de prueba 0 No acertado
Valor real tomada de las etiquetas:  3
-------------------------------------------------------------------------
Prediccion de la imagen de prueba 2 No acertado
Valor real tomada de las etiquetas:  4
-------------------------------------------------------------------------
Prediccion de la imagen de prueba 4 Acertado
Valor real tomada de las etiquetas:  4
-------------------------------------------------------------------------
Prediccion de la imagen de prueba 3 No acertado
Valor real tomada de las etiquetas:  5
-------------------------------------------------------------------------
Prediccion de la imagen de prueba 8 Acertado
Valor real tomada de las etiquetas:  8
-------------------------------------------------------------------------
Prediccion de la imagen de prueba 3 No acertado
Valor real tomada de las etiquetas:  9
-------------------------------------------------------------------------
    \end{Verbatim}

    \section{Conclusión}\label{conclusiuxf3n}

    \paragraph{A pesar de que se generó un modelo con unos resultados de
precisión del 92\% vemos que con unos datos que varian ya que los
digitos tienen trazos mas delgados solo acerto 3 de 8 es decir una
precisión menor al 50\% sin embargo, al aplicarlo sobre las imagenes que
se extrajeron al inicio acerto el 100\% de las
imagenes.}\label{a-pesar-de-que-se-generuxf3-un-modelo-con-unos-resultados-de-precisiuxf3n-del-92-vemos-que-con-unos-datos-que-varian-ya-que-los-digitos-tienen-trazos-mas-delgados-solo-acerto-3-de-8-es-decir-una-precisiuxf3n-menor-al-50-sin-embargo-al-aplicarlo-sobre-las-imagenes-que-se-extrajeron-al-inicio-acerto-el-100-de-las-imagenes.}

\paragraph{Un posible trabajo futuro seria recopilar e incluir al
conjunto de datos digitos manuscritos mas delgados con el fin de que el
conjunto de datos aumente su variedad alimentando asi los modelos con
mas cantidad de
datos.}\label{un-posible-trabajo-futuro-seria-recopilar-e-incluir-al-conjunto-de-datos-digitos-manuscritos-mas-delgados-con-el-fin-de-que-el-conjunto-de-datos-aumente-su-variedad-alimentando-asi-los-modelos-con-mas-cantidad-de-datos.}

    \begin{tcolorbox}[breakable, size=fbox, boxrule=1pt, pad at break*=1mm,colback=cellbackground, colframe=cellborder]
\prompt{In}{incolor}{ }{\boxspacing}
\begin{Verbatim}[commandchars=\\\{\}]

\end{Verbatim}
\end{tcolorbox}


    % Add a bibliography block to the postdoc
    
    
    
\end{document}
